\documentclass{article}
\usepackage{amsmath,amssymb,amsthm,latexsym,paralist}

\theoremstyle{definition}
\newtheorem{problem}{Problem}
\newtheorem*{solution}{Solution}
\newtheorem*{resources}{Resources}

\newcommand{\name}[2]{\noindent\textbf{Name: #1}\hfill \textbf{Section: #2}}
\newcommand{\honor}{\noindent On my honor, as an Aggie, I have neither
  given nor received any unauthorized aid on any portion of the
  academic work included in this assignment. Furthermore, I have
  disclosed all resources (people, books, web sites, etc.) that have
  been used to prepare this homework. \\[2ex]
 \textbf{Signature:} \underline{\hspace*{10cm}} }
 
\newcommand{\checklist}{\noindent\textbf{Checklist:}
\begin{compactitem}[$\Box$] 
\item Did you type in your name and section? 
\item Did you disclose all resources that you have used? \\
(This includes all people, books, websites, etc.\ that you have consulted.)
\item Did you sign that you followed the Aggie Honor Code? 
\item Did you solve all problems? 
\item Did you submit the .tex and .pdf files of your homework to the correct link on eCampus?
\item Did you submit a signed hardcopy of the pdf file in class? 
\end{compactitem}
}

\newcommand{\problemset}[1]{\begin{center}\textbf{Problem Set #1}\end{center}}
\newcommand{\duedate}[2]{\begin{quote}\textbf{Due dates:} Electronic
    submission of \textsl{yourLastName-yourFirstName-hw3.tex} and 
    \textsl{yourLastName-yourFirstName-hw3.pdf} files of this homework is due on
    \textbf{#1} on \texttt{http://ecampus.tamu.edu}. You will see two separate links
    to turn in the .tex file and the .pdf file separately. Please do not archive or compress the files.  
    A signed paper copy of the pdf file is due on \textbf{#2} \textsl{at the beginning of class}.
    \textbf{If any of the three submissions are missing, your work will not be graded.}\end{quote} }

\newcommand{\N}{\mathbf{N}}
\newcommand{\R}{\mathbf{R}}
\newcommand{\Z}{\mathbf{Z}}


\begin{document}
\vspace*{-15mm}
\begin{center}
{\large
CSCE 222 [Sections 502, 503] Discrete Structures for Computing\\[.5ex]
Spring 2017 -- Hyunyoung Lee\\}
\end{center}
\problemset{3}
\duedate{Friday, 2/10/2017 before 11:00 a.m.}{Friday, 2/10/2017}
\name{Joseph Martinsen}{503}
\begin{resources} (All people, books, articles, web pages, etc.\ that
  have been consulted when producing your answers to this homework.)
  https://www.youtube.com/watch?v=Uzlj6N5OYcM
\end{resources}
\honor

\smallskip

\begin{problem} Section 2.1, Exercise 24, page 126. \textsl{Explain.}
\end{problem}
\begin{solution} Cardinality of set $|A| = k$ and of power set is $|P| = 2^k$ given by lecture and book
\begin{enumerate}[a)]
  \item The power set of every set contains the empty set. A power set then cannot be empty.$\therefore$ this is not a power set
  
  \item Consider in this case $A = \{a\}$\\
  $|A| = 1$\\
  $|P(A)| = 2^1 = 2$ consisting of $\{a\}$ and the empty set. $\therefore$ the given $\{\emptyset, \{a\}\}$ is the power set of $\{a\}$
  
  \item The given power set contains 3 elements. There are no number of elements $k$ that satisfy $2^k = 3$ where $|A| = k$ $\therefore$ this cannot be a power set
  
  \item Consider in this case $A = \{a,b\}$ \\
  $|A| = 2$ so it follows that power set must contain $2^2 = 4$ elements. \\
  These elements are $\{a,b\}, \{a\}, \{b\}, \emptyset \therefore$ the given must be the power set of $\{a,b\}$
\end{enumerate}
\end{solution}

\begin{problem} Show that a set which is a subset of every set must be the empty set.
\end{problem}
\begin{solution}
$A \subset B$ if and only if every element of $A$ is also a member of $B$. If $A$ is the empty set, meaning that A has no members within its set, all 0 members of $A$ are also all members of $B$, no mater what members of $B$ has. \\$\therefore$ the empty set is a subset of every set.
\end{solution}

\begin{problem} Let $A$ and $B$ be sets. Show that $P(A)=P(B)$ implies $A=B$.
\end{problem}
\begin{solution}
\begin{align*}
  (P(A)=P(B))&\rightarrow(A=B) \\
\end{align*}
% Assume $A \neq B$. There must be a be an $x \in A$ that is $x \notin B$. It directly follows that $\{x\} \in P(A)$ and $\{x\} \notin P(B)$. This results
\begin{align*}
  P(A) &= P(B) \\
  \forall C(C \in P(A) &\land C \in P(B)) &\textbf{Universal instantiation} \\
  \forall C(C \subseteq A &\land C  B) \\
  A &= B
\end{align*}
\end{solution}

\begin{problem} (\textbf{20 Points}) Section 2.2, Exercise 16, page 136.
\end{problem}
\begin{solution}\ \\
\begin{enumerate}[a)]
  \item
  \begin{align*}
    A \cap B \equiv& \{x | x \in (A \cap B) \} &\textbf{By definition} \\
    \equiv& x \text{ must be members of $A$ and $B$}\\
    \subseteq& x \text{ is a member of $A$} &\textbf{This is a proper subset of the previous}\\
    \subseteq& A &\textbf{By generalization} \qed
  \end{align*}
  \item
  \begin{align*}
    A &= \{x \in A \} & \textbf{By instatiation} \\
    &\subseteq \{x \in A \lor x \in B\} &\textbf{By addition} \\
    &\subseteq \{x \in A \cup B \} &\textbf{By combination} \\
    &\subseteq \{A \cup B \} &\textbf{By generalization} \qed
  \end{align*}
  \item
  \begin{align*}
    A - B &= \{x \in (A - B) \} \\
    &= \{x \in A \land x \notin B \} \\
    &\subseteq \{x \in A \} \\
    &\subseteq A \qed
    \end{align*}
    \item
    \begin{align*}
      A \cap (B - A) &= A \cap \{x \in B \land x \notin A \} \\
      &= A \cap \{x \in B \land x \in \bar A \} \\
      &= A \cap (B \cap \bar A) \\
      &= (A \cap \bar A) \cap B &\textbf{By communative property} \\
      &= \emptyset \cap B \\
      &= \emptyset
    \end{align*}
    \item
    \begin{align*}
      A \cup (B - A) &= A \cup \{x \in B \land x \notin A \} \\
      &= A \cup \{x \in B \land x \in \bar A \} \\
      &= A \cup (B \cap \bar A) \\
      &= (A \cup B) \cap (A \cup \bar A) &\textbf{By distribution} \\
      &= (A \cup B) \cap U &\textbf{$A \cup \bar A =$ Power set} \\
      &= A \cup B &\textbf{By $A \cap U = A$}
    \end{align*}
\end{enumerate}
\end{solution}

\begin{problem} Show that $A\cap (B - C) = (A\cap B) - C$.
[Hint: Start out by expanding the definition of $(B - C)$.]
\end{problem}
\begin{solution}
\begin{align*}
  A \cap (B - C) &= A \cap (x \in B | x \notin C ) \\
  &= A \cap (B \cap \bar C) \\
  &= (A \cap B) \cap \bar C &\textbf{By communative property} \\
  &= \{x \in (A \cap B) | x \in \bar C \} \\
  &= \{x \in (A \cap B) | x \notin C \} \\
  &= (A \cap B) - C \qed
\end{align*}
\end{solution}

\begin{problem} Section 2.3, Exercise 12, page 153. \textsl{Explain.} 
\end{problem}
\begin{solution}
\begin{enumerate}[a)]
  \item 
  \begin{align*}
    f(a) &= f(b) \\
    a-1 &= b-1 \\
    a &= b
  \end{align*}
  $\therefore$ this function is one-to-one
  
  \item
  \begin{align*}
    f(a) &= f(b) \\
    a^2 + 1 &= b^2 + 1 \\
    a^2 &= b^2 \\
    a &= \pm b
  \end{align*}
  $\therefore$ this function is NOT one-to-one
  \item
  \begin{align*}
    f(a) &= f(b) \\
    a^3 &= b^3 \\
    a &= b \\
  \end{align*}
  For all this holds $\therefore$ this function is one-to-one
  \item
  \begin{align*}
    f(2) = \left \lceil \dfrac{2}{2} \right\rceil = 1 \\
    f(1) = \left \lceil \dfrac{1}{2} \right\rceil = 1
  \end{align*}
  $\therefore$ this function is not one-to-one
\end{enumerate}
\end{solution}

\begin{problem} Section 2.3, Exercise 14 a), b), c) and d), page 153. \textsl{Explain.}
\end{problem}
\begin{solution}\ \\
\begin{enumerate}[a)]
  \item
  \begin{equation}
    k = 2m - n
  \end{equation}
  $k$ is always $k \in \mathbb{Z}$ because any linear transformations of integers is an integer
  \begin{equation}
    f(m,n) = 2m - s = p
  \end{equation}
  $\therefore$ this function is onto
  
  \item
  There does not exists any integer solution that satisfy $m^2 - n^2 = 2 \therefore$ this function is not onto
  \item
  \begin{equation}
    k = m + n + 1
  \end{equation}
  $k$ is always $k \in \mathbb{Z}$ because any linear transformations of integers is an integer
  \begin{equation}
    f(a,b) = a + b + 1 = p
  \end{equation}
  $\therefore$ this function is onto for any integer value
  \item
  If $m = 0$, $f(0,n) = - |n|$. This equation covers all negative integer values $<0$. If $n=0$, $f(m,0) = |m|$. This equation covers all positive integer values. If $m=n$, $f(m,m) = |m| - |m| = 0$. This covers 0. $\therefore$ this function is onto for any integer value 
  
\end{enumerate}
\end{solution}

\begin{problem} Section 2.3, Exercise 50, page 154. 
\end{problem}
\begin{solution}
\begin{align*}
  \left \lceil x + m \right \rceil = \lceil x \rceil + m \\
  \text{Suppose }\lceil x \rceil = n \\
  \text{It follows } n-1 < x \leq n \\
  \text{Add $m$, } n-1 +m < x + m \leq n + m \\
  \text{By the ceil function definition } \lceil x + m \rceil = n + m \\
  \text{By $\lceil x \rceil = n$, }\lceil x + m \rceil = \lceil x \rceil + m \qed
\end{align*}
\end{solution}

\begin{problem} Section 2.3, Exercise 58, page 154. \textsl{Explain.}
\end{problem}
\begin{solution}
\begin{enumerate}[a)]
  \item There are 8 bits in a byte. $\left \lceil \dfrac{4}{8} \right \rceil = 1$ To encode only 4 bits, only 1 byte is needed
  $\left \lceil \dfrac{10}{8} \right \rceil = 2$ To encode a 10 bits of data, 2 bytes are needed
  \item $\left \lceil \dfrac{500}{8} \right \rceil =  63$ bytes are needed to encode 500 bits
  \item $\left \lceil \dfrac{3000}{8} \right \rceil = 375$ Exactly 375 bytes are needed to encode 3000 bits
\end{enumerate}
\end{solution}

\begin{problem} (\textbf{Extra credit 10 points}) Prove that 
$$ \left\lceil {\left\lceil \displaystyle\frac{x}{2} \right\rceil} \big{/} {2} \right\rceil = 
\left\lceil \displaystyle\frac{x}{4} \right\rceil$$ holds for all real numbers $x$.
\end{problem}
\begin{solution}
\end{solution}

\goodbreak
\checklist
\end{document}

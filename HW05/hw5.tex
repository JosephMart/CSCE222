\documentclass{article}
\usepackage{amsmath,amssymb,amsthm,latexsym,paralist}

\theoremstyle{definition}
\newtheorem{problem}{Problem}
\newtheorem*{solution}{Solution}
\newtheorem*{resources}{Resources}

\newcommand{\name}[2]{\noindent\textbf{Name: #1}\hfill \textbf{Section: #2}}
\newcommand{\honor}{\noindent On my honor, as an Aggie, I have neither
  given nor received any unauthorized aid on any portion of the
  academic work included in this assignment. Furthermore, I have
  disclosed all resources (people, books, web sites, etc.) that have
  been used to prepare this homework. \\[2ex]
 \textbf{Signature:} \underline{\hspace*{10cm}} }
 
\newcommand{\checklist}{\noindent\textbf{Checklist:}
\begin{compactitem}[$\Box$] 
\item Did you type in your name and section? 
\item Did you disclose all resources that you have used? \\
(This includes all people, books, websites, etc.\ that you have consulted.)
\item Did you sign that you followed the Aggie Honor Code? 
\item Did you solve all problems? 
\item Did you submit the .tex and .pdf files of your homework to the correct link on eCampus?
\item Did you submit a signed hardcopy of the pdf file in class? 
\end{compactitem}
}

\newcommand{\problemset}[1]{\begin{center}\textbf{Problem Set #1}\end{center}}
\newcommand{\duedate}[2]{\begin{quote}\textbf{Due dates:} Electronic
    submission of \textsl{yourLastName-yourFirstName-hw5.tex} and 
    \textsl{yourLastName-yourFirstName-hw5.pdf} files of this homework is due on
    \textbf{#1} on \texttt{http://ecampus.tamu.edu}. You will see two separate links
    to turn in the .tex file and the .pdf file separately. Please do not archive or compress the files.  
    A signed paper copy of the pdf file is due on \textbf{#2} \textsl{at the beginning of class}.
    \textbf{If any of the three submissions are missing, your work will not be graded.}\end{quote} }

\newcommand{\N}{\mathbf{N}}
\newcommand{\R}{\mathbf{R}}
\newcommand{\Z}{\mathbf{Z}}


\begin{document}
\vspace*{-15mm}
\begin{center}
{\large
CSCE 222 [Sections 502, 503] Discrete Structures for Computing\\[.5ex]
Spring 2017 -- Hyunyoung Lee\\}
\end{center}
\problemset{5}
\duedate{Friday, 3/3/2017 before 11:00 a.m.}{Friday, 3/3/2017}
\name{Joseph Martinsen}{506}
\begin{resources} (All people, books, articles, web pages, etc.\ that
  have been consulted when producing your answers to this homework.)
\end{resources}
\honor

\smallskip

\begin{problem} (10 points) 
Section 2.4, Exercise 6 (b)--(f), pages 167--168
\end{problem}
\begin{solution}\ \\
\begin{enumerate}[a)]
  \setcounter{enumi}{1}
  \item
                               3
                               6
                               10
                               15
                               21
                               28
                               36
                               45
                               55
  
  \item
                                1
                               5
                               19
                               65
                              211
                              665
                              2059
                              6305
                             19171
                             58025

  \item
                                 1
                               1
                               1
                               2
                               2
                               2
                               2
                               2
                               3
                               3

  \item
                               1
                               5
                               6
                               11
                               17
                               28
                               45
                               73
                              118
                              191

  \item
  1 3 7 15 31 63 127 255 511 1023
\end{enumerate}
\end{solution}

\begin{problem} (10 points)
Section 2.4, Exercise 10, page 168
\end{problem}
\begin{solution}
\begin{enumerate}[a)]\ \\
  \item 
                               -1
                               -2
                               -4
                               -8
                              -16
                              -32

  \item
                                2
                               -1
                               -3
                               -2
                               1
                               3

  
  \item
                               1
                               3
                               27
                              2187
                            14348907
                        617673396283947

  
  \item
                              -1
                               0
                               1
                               3
                               13
                               74

  
  \item
                               1
                               1
                               2
                               2
                               1
                               1

  

\end{enumerate}
\end{solution}

\begin{problem} (20 points)
Section 2.4, Exercise 16 (a)--(e), page 168
\end{problem}
\begin{solution}\ \\
\begin{enumerate}[a)]
  \item 
  \begin{align*}
    a_n &= -a_{n-1} \; a_0 = 5 \\
    a_1 &= -5 \\
    a_2 &=  5 \\
    a_3 &= -5 \\
    a_4 &= 5 \\
    &\vdots \\
    a_n &= 5(-1)^n
  \end{align*}
  \item 
  \begin{align*}
    a_n &= a_{n-1} + 3 \; a_0 = 1 \\
    a_1 &=  1 + 3 = 4 \\
    a_2 &=  4 + 3 = 7 \\
    a_3 &=  7 + 3 = 10 \\
    &\vdots \\
    a_n &= 3n + 1
  \end{align*}
  \item 
  \begin{align*}
    a_n &= a_{n-1} - n \; a_0 = 4 \\
    a_1 &= 4 - 1  \\
    a_2 &= 4 - 1 - 2 \\
    a_3 &= 4 - 1 - 2 - 3 \\
    a_4 &= 4 - 1 - 2 - 3 - 4 \\
    &\vdots \\
    a_n & = 4 - \sum_{i=1}^n i \\
        a_n & = 4 - \dfrac{n(n+1)}{2}
  \end{align*}
  \item 
  \begin{align*}
        a_n &= 2a_{n-1} - 3 \; a_0 = -1 \\
    a_1 &= 2a_0 - 3 \\
    a_2 &= 2(2a_0 - 3) - 3 \\
    a_3 &= 2(2(2a_0 - 3) - 3) - 3 \\
    a_3 &= 2^3 a_0 - 4 \cdot 3 - 2 \cdot 3 - 3\\
    a_3 &= 2^3 a_0 - 2^2 \cdot 3 - 2^1 \cdot 3 - 2^0 \cdot 3\\
    &\vdots \\
    a_n &= 2^n - 3\dfrac{(2^n - 1)}{2-1} \\
    a_n &= 2^n -3 \cdot 2^n - 3
  \end{align*}
  \item 
  \begin{align*}
        a_n &= (n+1)a_{n-1} \; a_0 = 2 \\
    a_n &= (n+1)n a_{n-2} \\
    a_n &= (n+1)(n)(n-1) a_{n-3} \\
    a_n &= (n+1)(n)(n-1)(n-2) a_{n-4} \\
    a_n &= (n+1)(n)(n-1)(n-2) \dots (n-(n) + 1)  a_{n-n} \\
    a_n &= (n+1)(n)(n-1)(n-2) \dots 2 \\
    a_n &= 2(n+1)!
  \end{align*}
  
\end{enumerate}
\end{solution}

\begin{problem} (10 points)
Section 2.4, Exercise 34, page 169
\end{problem}
\begin{solution} \textit{does not ask to show}
\begin{enumerate}[a)]
  \item 3
  \item 78
  \item 9
  \item 180
\end{enumerate}
\end{solution}

\noindent
\textsl{Before attempting the problems below on proof-by-induction, make sure that 
you have carefully read Section 5.1.}

\begin{problem} (10 points) \textsl{Prove by mathematical induction} that
$$\sum_{i=0}^n 3^i = \frac{3^{n+1}-1}{2}$$
holds for every non-negative integer $n$.
\end{problem}
\begin{solution}
\end{solution}

\begin{problem} (15 points)
Section 5.1, Exercise 8, page 329 (use mathematical induction)
\end{problem}
\begin{solution} 
\end{solution}

\begin{problem} (15 points)
Section 5.1, Exercise 24, page 330 (use mathematical induction)
\end{problem}
\begin{solution} 
\end{solution}

\begin{problem} (10 points)
Section 5.1, Exercise 32, page 330 (use mathematical induction)
\end{problem}
\begin{solution} 
\end{solution}

\begin{problem} (\textbf{Extra credit 10 points})
Section 5.1, Exercise 14, page 330 (use mathematical induction)
\end{problem}
\begin{solution} 
\end{solution}

\goodbreak
\checklist
\end{document}

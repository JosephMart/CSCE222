\documentclass{article}
\usepackage{amsmath,amssymb,amsthm,latexsym,paralist,wasysym}

\theoremstyle{definition}
\newtheorem{problem}{Problem}
\newtheorem*{solution}{Solution}
\newtheorem*{resources}{Resources}

\newcommand{\name}[2]{\noindent\textbf{Name: #1}\hfill \textbf{Section: #2}}
\newcommand{\honor}{\noindent On my honor, as an Aggie, I have neither
  given nor received any unauthorized aid on any portion of the
  academic work included in this assignment. Furthermore, I have
  disclosed all resources (people, books, web sites, etc.) that have
  been used to prepare this homework. \\[2ex]
 \textbf{Signature:} \underline{\hspace*{10cm}} }
 
\newcommand{\checklist}{\noindent\textbf{Checklist:}
\begin{compactitem}[$\Box$] 
\item Did you type in your name and section? 
\item Did you disclose all resources that you have used? \\
(This includes all people, books, websites, etc.\ that you have consulted.)
\item Did you sign that you followed the Aggie Honor Code? 
\item Did you solve all problems? 
\item Did you submit the .tex and .pdf files of your homework to the correct link on eCampus?
\item Did you submit a signed hardcopy of the pdf file in class? 
\end{compactitem}
}

\newcommand{\problemset}[1]{\begin{center}\textbf{Problem Set #1}\end{center}}
\newcommand{\duedate}[2]{\begin{quote}\textbf{Due dates:} Electronic
    submission of \textsl{yourLastName-yourFirstName-hw6.tex} and 
    \textsl{yourLastName-yourFirstName-hw6.pdf} files of this homework is due on
    \textbf{#1} on \texttt{http://ecampus.tamu.edu}. You will see two separate links
    to turn in the .tex file and the .pdf file separately. Please do not archive or compress the files.  
    A signed paper copy of the pdf file is due on \textbf{#2} \textsl{at the beginning of class}.
    \textbf{If any of the three submissions are missing, your work will not be graded.}\end{quote} }

\newcommand{\N}{\mathbf{N}}
\newcommand{\R}{\mathbf{R}}
\newcommand{\Z}{\mathbf{Z}}


\begin{document}
\vspace*{-15mm}
\begin{center}
{\large
CSCE 222 [Sections 502, 503] Discrete Structures for Computing\\[.5ex]
Spring 2017 -- Hyunyoung Lee\\}
\end{center}
\problemset{6}
\duedate{Monday, 3/20/2017 before 11:00 a.m.}{Monday, 3/20/2017}
\name{Joseph Martinsen}{503}
\begin{resources} (All people, books, articles, web pages, etc.\ that
  have been consulted when producing your answers to this homework.)
\end{resources}
\honor

\smallskip

\begin{problem} (5 points) 
Section 5.1, Exercise 50, page 331
\end{problem}
\begin{solution} \ \\
In the basis step of this proof, there is an error. $$\displaystyle \dfrac{(1 + \frac{1}{2})^2}{2} = \left(\dfrac{3}{2} \right)^2 \dfrac{1}{2} = \dfrac{9}{8}$$ and $$\displaystyle \sum_{i=1}^1 i = 1$$ are not true or equal as stated.
\end{solution}

\begin{problem} (15 points) 
Section 5.2, Exercise 4, pages 341--342
\end{problem}
\begin{solution}\ \\
\begin{enumerate}[a)]
  \item \ \\
  \underline{\textbf{Basis Step}} \\
  $P(18):  1 \cdot 4$ cent stamps and $2 \cdot 7$ cent stamps \\
  $P(19):  3 \cdot 4$ cent stamps and $1 \cdot 7$ cent stamps \\
  $P(20):  5 \cdot 4$ cent stamps and $0 \cdot 7$ cent stamps \\
  $P(21):  0 \cdot 4$ cent stamps and $3 \cdot 7$ cent stamps
  
  \item
  The inductive hypothesis of this proof is as follows. Any $j \cent $ postage stamp can be made from a combination of $4\cent$ and $7\cent$ stamps for all $j$ stamps with $ 18 \leq k$. where $k \ge 21$.
  
  \item
  In order to prove the inductive step, show that $k+1 \cent$ stamps can be made from $4\cent$ and $7 \cent$ stamps.
  
  \item
  With $k \ge 21$, $P(k-3)$ has been shown to be true by the inductive hypothesis. Adding another $4 \cent$ stamp, it must follow that $P(k+1)$ is also true. 
  
  \item
  With the basis step and the inductive step shown to be valid, $P$ has been proved to hold for $n \ge 21$ by strong induction.
\end{enumerate}
\end{solution}

\begin{problem} (10 points)
Section 5.2, Exercise 12, page 342
\end{problem}
\begin{solution} \ \\
% \textbf{Assume} $f_0 - f_1 + f_2 - \dots - f_{2n-1} + f_{2n} = f_{2n-1} - 1 $ for $n\ge 1$ \\
\textbf{\underline{Basis Step}: Show }$P(0), P(1), P(2), P(3)$ \\
\begin{align*}
2^0 = 1 \\
2^1 = 2 \\
2^2 = 4 \\
2^3 = 8
\end{align*}
$\therefore P(0), P(1), P(2), P(3)$ holds \\
\textbf{\underline{Inductive Step}} Let $k \ge 1$, assume the claim holds for all $n$ where $1 \le n \le k$ \\ \ \\
\textbf{Case 1:} Assume $k+1$ is even. Then $\dfrac{k+1}{2}$ is an integer between 1 and $k$. The claim holds for $\dfrac{k+1}{2}$ by the strong induction hypothesis.
\begin{align*}
\dfrac{k+1}{2} &= 2m_1 + 2m_2 + \dots + 2 m_{\mathbb{L}} &\textbf{By Strong I.H.} \\
\dfrac{k+1}{2} &= 2(m_1 + 1) + 2(m_2+1) + \dots + 2 (m_{\mathbb{L}} + 1)
\end{align*}
$\therefore$ \textbf{the statement holds for when $k+1$ is even} \\ \ \\
\textbf{Case 2:} Assume $k+1$ is odd. By Strong Induction Hypothesis, the claim holds for $k$.
\begin{align*}
  k  &= 2m_1 + 2m_2 + \dots + 2 m_{\mathbb{L}} &\textbf{By Strong I.H.} \\
  k + 1 &= 1 + 2m_1 + 2m_2 + \dots + 2 m_{\mathbb{L}} \\
  k + 1 &= 2^0 + 2m_1 + 2m_2 + \dots + 2 m_{\mathbb{L}}
\end{align*}
$\therefore$ \textbf{the statement holds for when $k+1$ is odd} \\ \ \\
$\therefore$ \textbf{the statement holds by proof by Strong Induction}
\end{solution}

\begin{problem} (10 points)
Section 5.2, Exercise 30, page 344
\end{problem}
\begin{solution}\ \\
The basis step shows that $a^0 = 1$. The correct induction hypothesis would be $a^0 = 1$ not $a^j = 1$. The incorrect inductive hypothesis is used in this proof.
\end{solution}

\begin{problem} (30 points)
Section 5.3, Exercise 6, page 357.
To prove that your formula is valid, use mathematical induction.
For the invalid recursive definitions, explain why they are invalid.
\end{problem}
\begin{solution} \ \\
\begin{enumerate}[a)]
  \item $f(0) = 1 \quad f(1) = -1 \quad f(2) = 1$. It is apparent that the closed form solution is $f(n) = (-1)^n$ \\
  \textbf{\underline{Basis Step}: Show }$P(1)$ \\
  \begin{align*}
  P(1) &= f(1) = -f(0) = -1 \\
  P(1) &= f(1) = (-1)^1 = -1
  \end{align*}
  $\therefore P(2)$ holds \\
  \textbf{\underline{Inductive Step}: Show } $P(k) \rightarrow P(k+1)$ \\
  \textbf{Assume} $P(k)$ \textbf{for arbitrary $k \ge  1$}: $f(k) = (-1)^k$ \\
  \textbf{Show P(k+1):}
  \begin{align*}
    P(k+1) &= -f(k+1-1) \\
    &= (-1) f(k) \\
    &= (-1)^1 (-1)^k &\textbf{By Inductive Hypothesis}\\
    &= (-1)^{k+1}
  \end{align*}
  $\therefore P(k) \rightarrow P(k+1)$ \textbf{holds}\\
  $\therefore$ \textbf{the statement holds for all $n \ge 1 $ by mathematical induction}
  
  \item $f(0) = 1 \quad f(1) = 0 \quad f(2) = 2 \quad f(3) = 2 \quad f(4) = 0 \quad f(5) = 4$ \\
  $2^n $
  
  \item It is not well defined because each $n \ge 2$ is dependent on the the value of $f(n+1)$ which is unknown.
  
  \item It is not well defined because the base case $f(1)$ is given as $f(1) = 1$ but also for $n \ge 1$ it is also given that $f(1) = 2f(1-1) = 2f(0) = 0$. Because of 2 different definitions for the same value, this is not well defined.
  
  \item $ f(0) = 2 \quad f(1) = f(0) = 2 \quad f(2) = 2f(0) = 4 \quad f(3) = f(2) = 4 \quad f(4) = 2f(3) = 8 \quad f(5) = 8 \quad f(6) = 16 \quad f(7) = 16$. A closed form solution is given by $f(n) = 2^{\lfloor (n+1)/2 \rfloor}$
  \textbf{\underline{Basis Step}: Show }$P(1)$ \\
  \begin{align*}
  P(1) &= f(1) = f(0) = 2 \\
  P(1) &= f(1) = 2^{\lfloor (1+1)/2 \rfloor} = 2^1 = 2
  \end{align*}
  $\therefore P(1)$ holds \\
  \textbf{\underline{Inductive Step}: Show } $P(k) \rightarrow P(k+1)$ \\
  \textbf{Assume} $P(k)$ \textbf{for arbitrary $k \ge  1$}: $f(k) = 2^{\lfloor (k+1)/2 \rfloor}$ \\
  \textbf{Show P(k+1):} \\
  \textbf{Case 1:} assume k+1 is even
  \begin{align*}
    P(k+1) &= 2f(k+1 - 2) \\
    &= 2f(k - 1) \\
    &= 2 \cdot 2^{\lfloor (k - 1 +1)/2 \rfloor} &\textbf{By Inductive Hypothesis}\\
    &= 2^{\lfloor 1 + k/2 \rfloor} \\
    &= 2^{\lfloor (k+2)/2 \rfloor} \\
    &= 2^{\lfloor ((k+1) + 1)/2 \rfloor}
  \end{align*}
  $\therefore P(k) \rightarrow P(k+1)$ \textbf{holds} when $k+1$ is even\\
  \textbf{Case 2:} assume $k+1$ is odd
  \begin{align*}
    P(k+1) &= f(k+1 - 1) \\
    &= f(k) &\textbf{This holds by the Inductive Hypothesis}
  \end{align*}
  $\therefore$ \textbf{the statement holds for all $n \ge 1 $ by mathematical induction}
\end{enumerate}
\end{solution}

\begin{problem} (15 points)
Section 5.3, Exercise 16, page 358 (use mathematical induction)
\end{problem}
\begin{solution}  \ \\
$f_n$ is the $n^{th}$ Fibonacci number. \\
\textbf{Assume} $f_0 - f_1 + f_2 - \dots - f_{2n-1} + f_{2n} = f_{2n-1} - 1 $ for $n\ge 1$ \\
\textbf{\underline{Basis Step}: Show }$P(1)$ \\
\begin{align*}
P(1) &= f_0 - f_1 + f_2 = 0 - 1 + 1 = 0 \\
P(1) &= f_{2(1) - 1} - 1 = f_1 - 1 = 1 - 1 = 0
\end{align*}
$\therefore P(1)$ holds \\
\textbf{\underline{Inductive Step}: Show } $P(k) \rightarrow P(k+1)$ \\
\textbf{Assume} $P(k)$ \textbf{for arbitrary $k \ge  1$} \\
\textbf{Show P(k+1):} \\
\begin{align*}
  f_0 - f_1 + f_2 - \dots - f_{2(k+1)-1} + f_{2(k+1)} &= f_0 - f_1 + f_2 - \\
  & \dots - f_{2k-1} + f_{2k} - f_{2k+1} + f_{2k+2} \\
  &= f_{2k-1} - 1 - f_{2k+1} + f_{2k+2} &\textbf{By Inductive Hypothesis}\\
  &= f_{2k-1} - 1 + f_{2k} \\
  &= f_{2k+1} - 1 &\textbf{By Fib.}\\
  &= f_{2(k+1) - 1} - 1
\end{align*}
$\therefore P(k) \rightarrow P(k+1)$ \textbf{holds}\\
$\therefore$ \textbf{the statement holds for all $n \ge 1 $ by mathematical induction}
\end{solution}

\begin{problem} (15 points)
Section 5.3, Exercise 44, page 359
\end{problem}
\begin{solution} \ \\
$l(T)$ is the number of leaves of a full binary tree T.
$i(T )$ is the number of internal vertices of T \\
$P: l(T) = 1 + i(T)$ \\
\textbf{\underline{Basis Step}} \\
$P(0)$ is a tree with a single vertex. By the basis step of the given recursive definition, the single root is a leaf and there are no internal vertices.
\begin{align*}
  l(t) &= 1 \\
  i(t) &= 0 \\
  l(t) &= 1 + i(t) = 1
\end{align*}
$\therefore P(1)$ holds \\
Let $t$ be a tree smaller than $T$. Assume that the result is true for all $t$.
\textbf{\underline{Recursive Step:}}
Let $T_1$ be a left subtree and $T_2$ be a right subtree consisting of a root $r$. By the given definition $T = T_1 \cdot T_2$. It follows that
\begin{eqnarray}
  l(T) = l(T_1) + l(T_2)
\end{eqnarray}
The internal vertices of $T$ are the root $r$ of $T$ and the union of the
set of internal vertices of $T_1$ and the set of internal vertices
of $T_2$.
\begin{eqnarray}
  i(T) = i(T_1) + i(T_2) + 1
\end{eqnarray}
\begin{align*}
  l(T) &= l(T_1) + l(T_2) &\textbf{By (1)}\\
  &= i(T_1) + 1 + i(T_2) + 1 &\textbf{By Assumtion, $T_1,T_2$ are smaller than $T$}\\
  &= i(T) + 1 \qed &\textbf{By (2)}
\end{align*}
$\therefore$ the statment has been proven by structural induction.
\end{solution}

\begin{problem} (\textbf{Extra credit 10 points})
Section 5.3, Exercise 36, page 359
\end{problem}
\begin{solution} \ \\
Let $w_1 =$ abc and $w_2 =$ def. \\
\textbf{\underline{Basis Step}} \\
\begin{align}
  w_1 &= cba \\
  w_2 &= fed \\
  w_1 w_2 &= abc def \\
  (w_1 w_2)^r &= fedcba
\end{align}
\textbf{\underline{Recursive Step}} \\
\begin{align*}
  (w_2)^r (w_1)^r &= (fed) (cba) &\textbf{by (3)} \\
  &= (w_1w_2)^r &\textbf{by (6)} \\
\end{align*}
\end{solution}

\goodbreak
\checklist
\end{document}

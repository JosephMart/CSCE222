\documentclass{article}
\usepackage{amsmath,amssymb,amsthm,latexsym,paralist}
\usepackage{mathtools}
\DeclarePairedDelimiter\ceil{\lceil}{\rceil}
\DeclarePairedDelimiter\floor{\lfloor}{\rfloor}
\theoremstyle{definition}
\newtheorem{problem}{Problem}
\newtheorem*{solution}{Solution}
\newtheorem*{resources}{Resources}

\newcommand{\name}[2]{\noindent\textbf{Name: #1}\hfill \textbf{Section: #2}}
\newcommand{\honor}{\noindent On my honor, as an Aggie, I have neither
  given nor received any unauthorized aid on any portion of the
  academic work included in this assignment. Furthermore, I have
  disclosed all resources (people, books, web sites, etc.) that have
  been used to prepare this homework. \\[2ex]
 \textbf{Signature:} \underline{\hspace*{10cm}} }
 
\newcommand{\checklist}{\noindent\textbf{Checklist:}
\begin{compactitem}[$\Box$] 
\item Did you type in your name and section? 
\item Did you disclose all resources that you have used? \\
(This includes all people, books, websites, etc.\ that you have consulted.)
\item Did you sign that you followed the Aggie Honor Code? 
\item Did you solve all problems? 
\item Did you submit the .tex and .pdf files of your homework to the correct link on eCampus?
\item Did you submit a signed hardcopy of the pdf file in class? 
\end{compactitem}
}

\newcommand{\problemset}[1]{\begin{center}\textbf{Problem Set #1}\end{center}}
\newcommand{\duedate}[2]{\begin{quote}\textbf{Due dates:} Electronic
    submission of \textsl{yourLastName-yourFirstName-hw7.tex} and 
    \textsl{yourLastName-yourFirstName-hw7.pdf} files of this homework is due on
    \textbf{#1} on \texttt{http://ecampus.tamu.edu}. You will see two separate links
    to turn in the .tex file and the .pdf file separately. Please do not archive or compress the files.  
    A signed paper copy of the pdf file is due on \textbf{#2} \textsl{at the beginning of class}.
    \textbf{If any of the three submissions are missing, your work will not be graded.}\end{quote} }

\newcommand{\N}{\mathbf{N}}
\newcommand{\R}{\mathbf{R}}
\newcommand{\Z}{\mathbf{Z}}


\begin{document}
\vspace*{-15mm}
\begin{center}
{\large
CSCE 222 [Sections 502, 503] Discrete Structures for Computing\\[.5ex]
Spring 2017 -- Hyunyoung Lee\\}
\end{center}
\problemset{7}
\duedate{Monday, 3/27/2017 before class begins}{Monday, 3/27/2017}
\name{Joseph }{Martinsen}
\begin{resources} (All people, books, articles, web pages, etc.\ that
  have been consulted when producing your answers to this homework.)
\end{resources}
\honor

\smallskip

\begin{problem} (16 points) 
Section 6.1, Exercise 24, page 396
\end{problem}
\begin{solution} \ \\
How many positive integers between 1000 and 9999 inclusive
\begin{enumerate}[a)]
  \item are divisible by 9?
  $$\floor*{ \dfrac{9999}{9}} - \floor*{\dfrac{1000-1}{9}} = \floor*{\dfrac{9000}{9}} = 1000$$
  \item are even?
  $$\floor*{ \dfrac{9999}{2}} - \floor*{\dfrac{1000-1}{2}} = \floor*{\dfrac{9000}{2}} = 4500$$
  \item have distinct digits?
  $$9 \cdot 9 \cdot 8 \cdot 7 = 4536$$
  \item are not divisible by 3?
  $$9999-999 - \floor*{ \dfrac{9999}{3}} - \floor*{\dfrac{1000-1}{3}} = 1000 - \floor*{\dfrac{9000}{3}} = 6000$$
  \item are divisible by 5 or 7? \\
    Divisible by 5:
    $$\floor*{ \dfrac{9999}{5}} - \floor*{\dfrac{1000}{5}} = 1799$$
    Divisible by 7:
    $$\floor*{ \dfrac{9999}{7}} - \floor*{ \dfrac{1000}{7}} = 1286$$
    Divisible by 5 and 7:
    $$\floor*{ \dfrac{9999}{5\cdot7}} - \floor*{ \dfrac{1000}{5\cdot7}} = 257$$
    divisible by 5 or 7:
    $$1799+1286-257 = 2828$$
  \item are not divisible by either 5 or 7?
  $$9000- 2828 = 6172$$
  \item are divisible by 5 but not by 7?
  $$1799 - 257 = 1542$$
  \item are divisible by 5 and 7?
  $$257$$
\end{enumerate}
\end{solution}

\begin{problem} (16 points) 
Section 6.1, Exercise 32, page 397
\end{problem}
\begin{solution} \ \\
How many strings of eight uppercase English letters are
there
\begin{enumerate}[a)]
  \item if letters can be repeated?
  $$26^8$$
  \item if no letter can be repeated?
  $$\dfrac{26!}{18!}$$
  \item that start with X, if letters can be repeated?
  $$26^7$$
  \item that start with X, if no letter can be repeated?
  $$ \dfrac{25!}{18!}$$
  \item that start and end with X, if letters can be repeated?
  $$26^6$$
  \item that start with the letters BO (in that order), if letters can be repeated?
  $$26^6$$
  \item that start and end with the letters BO (in that order), if letters can be repeated?
  $$26^4$$
  \item that start or end with the letters BO (in that order), if letters can be repeated?
  $$2\cdot 26^6 - 26^4$$
\end{enumerate}
\end{solution}

\begin{problem} (10 points) 
Section 6.1, Exercise 46, page 397
\end{problem}
\begin{solution} \ \\
In how many ways can a photographer at a wedding arrange 6 people in a row from a group of 10 people, where
the bride and the groom are among these 10 people, if
\begin{enumerate}[a)]
  \item the bride must be in the picture?
  $$6 \cdot C(9,5) = 90720$$
  \item both the bride and groom must be in the picture?
  $$6 \cdot 5 \cdot C(8,4) = 50400$$
  \item exactly one of the bride and the groom is in the picture?
  $$6 \cdot C(8,5) + 6 \cdot C(8,5) = 80640$$
\end{enumerate}
\end{solution}

\begin{problem} (10 points) 
Section 6.1, Exercise 62, page 398.  Explain.
\end{problem}
\begin{solution}\ \\
% Suppose that p and q are prime numbers and that n = pq.
% Use the principle of inclusion–exclusion to find the num-
% ber of positive integers not exceeding n that are relatively
% prime to n.
Inclusion-Explusion Principle:
$$\mid P \cup Q \mid = \mid P \mid + \mid Q \mid - \mid P \cap Q \mid$$
WolframAlpha defines relatively prime as: Two integers are relatively prime if they share no common positive factors (divisors) except 1
\begin{align*}
  n &= pq\\
  \mid P \mid &= \floor*{\dfrac{n}{p}} \\
  \mid Q \mid &= \floor*{\dfrac{n}{q}} \\
  \mid P \cap Q \mid &= 1 \\
  \mid P \cup Q \mid &= \floor*{\dfrac{n}{p}} + \floor*{\dfrac{n}{q}} - 1
\end{align*}
Since $P \cup Q$ represent the values of $p$ and $q$ that go into $n$, it must be the case that $n - \mid P \cup Q \mid$ represent the number of values that are relatively prime to $n$.
$$ n - \mid P \cup Q \mid = n - \floor*{\dfrac{n}{p}} - \floor*{\dfrac{n}{q}} + 1$$
\end{solution}

\begin{problem} (10 points) 
Section 6.2, Exercise 12, page 405.  Explain.
\end{problem}
\begin{solution} \ \\
Let $p(n)$ be $n \% 5$. The results of $p(n)$ are $0,1,2,3,4$. It follows that $\mid p(n) \mid = 5$. From the given statement, it must be the case that $(p(a_1),p(b_1)) = (p(a_2),p(b_2))$. There are $5\cdot 5 = 25$ different ordered pars of the from $(p(a),p(b))$. By the pigeonhole principle, 26 ordered pairs are needed to guarantee that the statement is satisfied.
\end{solution}

\begin{problem} (10 points) 
Section 6.2, Exercise 14, page 405.
\end{problem}
\begin{solution}
a) Show that if seven integers are selected from the first
10 positive integers, there must be at least two pairs
of these integers with the sum 11.
b) Is the conclusion in part (a) true if six integers are
selected rather than seven?
\begin{enumerate}[a)]
  \item The combination of numbers from the first 10 positive integers that result in 11 are $11 = 10+1=9+2=8+3=7+4=6+5$, a total of 5 combinations. Once 5 numbers have been selected that are not from this combination. The resulting 2 choices will complete the set.
  \item The statement from a) is false if there are only 6 numbers chosen. Only one can be guaranteed.
\end{enumerate}
\end{solution}

\begin{problem} (10 points) 
Section 6.3, Exercise 20, page 413.
\end{problem}
\begin{solution} 
\end{solution}

\begin{problem} (10 points) 
Section 6.3, Exercise 22 b), c), d), e), and f), page 414.
\end{problem}
\begin{solution} 
\end{solution}

\begin{problem} (10 points) 
Section 6.4, Exercise 12, page 421.
\end{problem}
\begin{solution} 
\end{solution}

\begin{problem} (\textbf{Extra credit 10 points})
Section 6.2, Exercise 46, page 407.  Prove the claim by contradiction 
(that is similar to the proof for the generalized pigeonhole principle shown 
in slide \#18 in the lecture slides counting.pdf).
\end{problem}
\begin{solution} 
\end{solution}

\goodbreak
\checklist
\end{document}

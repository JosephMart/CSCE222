\documentclass{article}
\usepackage{amsmath,amssymb,amsthm,latexsym,paralist,enumitem}

\theoremstyle{definition}
\newtheorem{problem}{Problem}
\newtheorem*{solution}{Solution}
\newtheorem*{resources}{Resources}

\newcommand{\name}[2]{\noindent\textbf{Name: #1}\hfill \textbf{Section: #2}}
\newcommand{\honor}{\noindent On my honor, as an Aggie, I have neither
  given nor received any unauthorized aid on any portion of the
  academic work included in this assignment. Furthermore, I have
  disclosed all resources (people, books, web sites, etc.) that have
  been used to prepare this homework. \\[2ex]
 \textbf{Signature:} \underline{\hspace*{8cm}} }
 
 
\newcommand{\checklist}{\noindent\textbf{Checklist:}
\begin{compactitem}[$\Box$] 
\item Did you type in your name and section? 
\item Did you disclose all resources that you have used? \\
(This includes all people, books, websites, etc.\ that you have consulted)
\item Did you sign that you followed the Aggie honor code? 
\item Did you solve all problems? 
\item Did you submit both the .tex and .pdf files of your homework separately 
to the correct link on eCampus?
\item Did you submit a signed hardcopy of the pdf file in class? 
\end{compactitem}
}

\newcommand{\problemset}[1]{\begin{center}\textbf{Problem Set #1}\end{center}}
\newcommand{\duedate}[2]{\begin{quote}\textbf{Due dates:} Electronic
    submission of \textsl{yourLastName-yourFirstName-hw8.tex} and 
    \textsl{yourLastName-yourFirstName-hw8.pdf} files of this homework is due on
    \textbf{#1} on \texttt{http://ecampus.tamu.edu}. You will see two separate links
    to turn in the .tex file and the .pdf file separately. Please do not archive or compress the files.  
    A signed paper copy of the pdf file is due on \textbf{#2} at the beginning of class.
    \textbf{If any of the three submissions are missing, your work will not be graded.}
    \textbf{Late submissions of any form will not be accepted.}\end{quote} }

\newcommand{\N}{\mathbf{N}}
\newcommand{\R}{\mathbf{R}}
\newcommand{\Z}{\mathbf{Z}}


\begin{document}
\vspace*{-18mm}
\begin{center}
{\large
CSCE 222 [Sections 502, 503] Discrete Structures for Computing\\[.5ex]
Spring 2017 -- Hyunyoung Lee\\}
\end{center}
\problemset{8}
\duedate{Monday, 4/10/2017 \textit{before} the beginning of class}{Monday, 4/10/2017}
\name{ Joseph }{Martinsen}
\begin{resources} (All people, books, articles, web pages, etc.\ that
  have been consulted when producing your answers to this homework)
  
  http://math.stackexchange.com/questions/881005/find-the-recurrence-relation-for-the-number-of-bit-strings-that-contain-the-stri
\end{resources}
\honor

\bigskip

\noindent
In this problem set, you will earn total $100+10$ (extra credit) points.

\begin{problem} (10 points)
Section 6.4, Exercise 38, page 422
\end{problem}
\begin{solution} \ \\
\begin{align*}
  n(n+1) \cdot 2^{n-1} &= n(n-1+2) \cdot 2^{n-2} \\
  &= n(n-1) \cdot 2^{n-2} + n2^{n-1}
\end{align*}
$n(n-1)\cdot 2^{n-2}$ corresponds to the number of ways of choosing a subset of $n$ elements when the two elements are different. \\
$n2^{n-1}$ corresponds to the number of ways of choosing a subset of set of $n$ elements when the 2 elements are the same. \\
$\therefore$ the number of ways to choose a subset of of a set of $n$ elements with 2 separate items that may or may not be differential-able is given in the statement. 
\end{solution}

\begin{problem} (10 points)
Section 8.1, Exercise 10, page 511
[Hint: Let $s_n$ denote the number of bit strings of length $n$ that
contain the string $01$. One of the initial conditions is $s_0 = 0$.] 
\end{problem}
\begin{solution} \ \\
\begin{enumerate}[label=(\alph*)]
  \item 
  \begin{align*}
   a_n&=a_{n-1}+2^{n-1}+2^{n-3}+ \cdots +2^{n-n} \\
   r &= \dfrac{1}{2} \\
   a_n &= a_{n-1} + 2^{n-1} - 1 &\textbf{by Geometric Summation}
  \end{align*}
  \item 
  $s_0 = 0 \qquad s_1 = 0$
  
  \item Using python I was able to compute the following:
\end{enumerate}

  
\end{solution}

\begin{problem} (10 points)
Section 8.1, Exercise 28, page 512. This problem has two parts as below. 
\end{problem}
\begin{solution} 
\ \\
a) (4 points) Show that the Fibonacci numbers satisfy \ldots
\ \\
\begin{align*}
  f_5 &= 5 f_1 + 3 f_0 \\
  f_6 &= 5f_2 + 3f_1 \\
  f_7 &= 5f_3 + 3f_2 \\
  f_8 &= 5f_4 + 3f_3 \\
  f_9 &= 
\end{align*}

b) (6 points) Use this recurrence relation to show that \ldots 
(prove by induction on $n$)
\end{solution}

\begin{problem} (10 points)
Section 8.1, Exercise 32 a), b), c) and d), page 512
\end{problem}
\begin{solution} 
\end{solution}

\begin{problem} ($5\times 8 \mbox{\,pts} = 40$ points)
Section 8.2, Exercise 4 a), b), c), d), and e), page 524.  
\textsl{For each subproblem, prove by induction that the closed form solution 
you found is correct.  Each subproblem is worth 8 points: 3 points for the closed 
form solution and 5 points for the \underline{correct} induction proof.}
\end{problem}
\begin{solution} 
\end{solution}

\begin{problem} (10 points)
Section 8.2, Exercise 8, pages 524--525
\end{problem}
\begin{solution} 
\end{solution}


\begin{problem} (10 points)
Section 8.4, Exercise 6 a), b), c), d) and e), page 549
\end{problem}
\begin{solution} 
\end{solution}

\begin{problem} (10 points)
Section 8.4, Exercise 8 a), b), c), d) and e), page 549
\end{problem}
\begin{solution} 
\end{solution}

\goodbreak
\checklist
\end{document}

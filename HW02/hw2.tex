\documentclass{article}
\usepackage{amsmath,amssymb,amsthm,latexsym,paralist}

\theoremstyle{definition}
\newtheorem{problem}{Problem}
\newtheorem*{solution}{Solution}
\newtheorem*{resources}{Resources}

\newcommand{\name}[2]{\noindent\textbf{Name: #1}\hfill \textbf{Section: #2}}
\newcommand{\honor}{\noindent On my honor, as an Aggie, I have neither
  given nor received any unauthorized aid on any portion of the
  academic work included in this assignment. Furthermore, I have
  disclosed all resources (people, books, web sites, etc.) that have
  been used to prepare this homework. \\[2ex]
 \textbf{Signature:} \underline{\hspace*{10cm}} }
 
\newcommand{\checklist}{\noindent\textbf{Checklist:}
\begin{compactitem}[$\Box$] 
\item Did you type in your name and section? 
\item Did you disclose all resources that you have used? \\
(This includes all people, books, websites, etc.\ that you have consulted.)
\item Did you sign that you followed the Aggie Honor Code? 
\item Did you solve all problems? 
\item Did you submit the .tex and .pdf files of your homework to the correct link on eCampus?
\item Did you submit a signed hardcopy of the pdf file in class? 
\end{compactitem}
}

\newcommand{\problemset}[1]{\begin{center}\textbf{Problem Set #1}\end{center}}
\newcommand{\duedate}[2]{\begin{quote}\textbf{Due dates:} Electronic
    submission of \textsl{yourLastName-yourFirstName-hw2.tex} and 
    \textsl{yourLastName-yourFirstName-hw2.pdf} files of this homework is due on
    \textbf{#1} on \texttt{http://ecampus.tamu.edu}. You will see two separate links
    to turn in the .tex file and the .pdf file separately. Please do not archive or compress the files.  
    A signed paper copy of the pdf file is due on \textbf{#2} \textsl{at the beginning of class}.
    \textbf{If any of the three submissions are missing, your work will not be graded.}\end{quote} }

\newcommand{\N}{\mathbf{N}}
\newcommand{\R}{\mathbf{R}}
\newcommand{\Z}{\mathbf{Z}}


\begin{document}
\vspace*{-15mm}
\begin{center}
{\large
CSCE 222 [Sections 502, 503] Discrete Structures for Computing\\[.5ex]
Spring 2017 -- Hyunyoung Lee\\}
\end{center}
\problemset{2}
\duedate{Friday, 2/3/2017 before 11:00 a.m.}{Friday, 2/3/2017}
\name{(type your name here)}{(type your section here)}
\begin{resources} (All people, books, articles, web pages, etc.\ that
  have been consulted when producing your answers to this homework.)
\end{resources}
\honor

\smallskip

\begin{problem} Section 1.4, Exercise 30, page 54. 
\end{problem}
\begin{solution}
\end{solution}

\begin{problem} Section 1.4, Exercise 36, page 55.
\end{problem}
\begin{solution}
\end{solution}

\begin{problem} Section 1.5, Exercise 28 b), c), e), and i), page 67. 
\textsl{Justify your answer or give a counterexample.}
\end{problem}
\begin{solution}
\end{solution}

\begin{problem} Section 1.5, Exercise 46, page 68.
\textsl{Justify your answer or give a counterexample.}
\end{problem}
\begin{solution}

\end{solution}

\begin{problem} Section 1.6, Exercise 6, page 78. 
\end{problem}
\begin{solution}
\end{solution}

\begin{problem} Section 1.6, Exercise 14 d), page 79.  
\end{problem}
\begin{solution}
\end{solution}

\begin{problem} Section 1.7, Exercise 18, page 91.
\end{problem}
\begin{solution}
\end{solution}

\begin{problem} Section 1.7, Exercise 22, page 91.  \textsl{Prove by contradiction.}
\end{problem}
\begin{solution}
\end{solution}

\begin{problem} Section 1.7, Exercise 24, page 91.  \textsl{Prove by contradiction.}
\end{problem}
\begin{solution}
\end{solution}

\begin{problem} Let $n>1$ be an integer. \textsl{Prove by contradiction} that 
if $n$ is a perfect square, then $n+3$ cannot be a perfect square. 
\end{problem}
\begin{solution}
\end{solution}

\goodbreak
\checklist
\end{document}

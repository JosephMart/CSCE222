\documentclass{article}
\usepackage{amsmath,amssymb,amsthm,latexsym,paralist}

\theoremstyle{definition}
\newtheorem{problem}{Problem}
\newtheorem*{solution}{Solution}
\newtheorem*{resources}{Resources}

\newcommand{\name}[2]{\noindent\textbf{Name: #1}\hfill \textbf{Section: #2}}
\newcommand{\honor}{\noindent On my honor, as an Aggie, I have neither
  given nor received any unauthorized aid on any portion of the
  academic work included in this assignment. Furthermore, I have
  disclosed all resources (people, books, web sites, etc.) that have
  been used to prepare this homework. \\[2ex]
 \textbf{Signature:} \underline{\hspace*{10cm}} }
 
\newcommand{\checklist}{\noindent\textbf{Checklist:}
\begin{compactitem}[$\Box$] 
\item Did you type in your name and section? 
\item Did you disclose all resources that you have used? \\
(This includes all people, books, websites, etc.\ that you have consulted.)
\item Did you sign that you followed the Aggie Honor Code? 
\item Did you solve all problems? 
\item Did you submit the .tex and .pdf files of your homework to the correct link on eCampus?
\item Did you submit a signed hardcopy of the pdf file in class? 
\end{compactitem}
}

\newcommand{\problemset}[1]{\begin{center}\textbf{Problem Set #1}\end{center}}
\newcommand{\duedate}[2]{\begin{quote}\textbf{Due dates:} Electronic
    submission of \textsl{yourLastName-yourFirstName-hw2.tex} and 
    \textsl{yourLastName-yourFirstName-hw2.pdf} files of this homework is due on
    \textbf{#1} on \texttt{http://ecampus.tamu.edu}. You will see two separate links
    to turn in the .tex file and the .pdf file separately. Please do not archive or compress the files.  
    A signed paper copy of the pdf file is due on \textbf{#2} \textsl{at the beginning of class}.
    \textbf{If any of the three submissions are missing, your work will not be graded.}\end{quote} }

\newcommand{\N}{\mathbf{N}}
\newcommand{\R}{\mathbf{R}}
\newcommand{\Z}{\mathbf{Z}}


\begin{document}
\vspace*{-15mm}
\begin{center}
{\large
CSCE 222 [Sections 502, 503] Discrete Structures for Computing\\[.5ex]
Spring 2017 -- Hyunyoung Lee\\}
\end{center}
\problemset{2}
\duedate{Friday, 2/3/2017 before 11:00 a.m.}{Friday, 2/3/2017}
\name{Joseph Martinsen}{503}
\begin{resources} (All people, books, articles, web pages, etc.\ that
  have been consulted when producing your answers to this homework.)
\end{resources}
\honor

\smallskip

\begin{problem} Section 1.4, Exercise 30, page 54. 
\end{problem}
\begin{solution}
\begin{enumerate}[a)]
  \ \\ $x \in \{1,2,3 \} \quad y \in \{1,2,3\}$
  \item
  \begin{align*}
    \exists x \; P(x,3) \equiv& P(1,3) \lor P(2,3) \lor P(3,3)
  \end{align*}
  
  \item
  \begin{align*}
    \forall y \; P(1,y) \equiv& P(1,1) \land P(1,2) \land P(1,3)
  \end{align*}
  \item
  \begin{align*}
    \exists y \; \neg P(2,y) \equiv& \neg P(2,1) \lor \neg P(2,2) \lor \neg P(2,3)
  \end{align*}
  \item
  \begin{align*}
    \forall x \; \neg P(x,2) \equiv \neg P(1,2) \land \neg P(2,2) \land \neg P(3,2)
  \end{align*}
\end{enumerate}
\end{solution}

\begin{problem} Section 1.4, Exercise 36, page 55.
\end{problem}
\begin{solution}
\begin{enumerate}[a)] \ \\
  \item When $x=1, \; 1^2 = 1$ and $x=0, \; 0^2 = 0$ \\
        $\therefore$ The counter examples are $x=0,1$
  \item When $x=\sqrt{2}, \; (\sqrt{2})^2 = 2$ and $x=-\sqrt{2}, \; (-\sqrt{2})^2 = 2$ \\
        $\therefore$ The counter examples are $x=\pm \sqrt{2}$
  \item When $x=0, \; |0| = 0$ it follows that $0 \ngtr 0$ \\
        $\therefore$ The counter example is $x=0$
\end{enumerate}
\end{solution}

\begin{problem} Section 1.5, Exercise 28 b), c), e), and i), page 67. 
\textsl{Justify your answer or give a counterexample.}
\end{problem}
\begin{solution} \ \\
b) $\forall x \exists y (x = y^2) $ is \textbf{false} for when $x = -1$ because there is no real number that satisfies $y^2 = -1$ \vspace{8pt} \\
c) $\exists x \forall y(xy = 0)$ is \textbf{true} for all $y$ when $ x=0$ \vspace{8pt}\\
e) $\forall x (x \neq 0 \rightarrow \exists y (xy = 1))$ for all $x$ that does not equal 0 there exists a real number $y$ that is equivalent to $\dfrac{1}{x}$ and it can be shown that $x \cdot \dfrac{1}{x} = 1 \therefore$ the statement is \textbf{true} \vspace{8pt}\\
i) $\forall x \exists y (x + y = 2 \land 2x - y = 1)$ \\
For when $x = 0$ the equation $x+y=2$ the solution for $y$ is 2. Also for when $x = 0$ the equation $2x-y=1$ the solution for $y$ is -1 \\
$\therefore$ the statement is \textbf{false} with $x=0$ as a counterexample.
\end{solution}

\begin{problem} Section 1.5, Exercise 46, page 68.
\textsl{Justify your answer or give a counterexample.}
\end{problem}
\begin{solution}\ \\ $\exists x \forall y \; (x \leq y^2)$
\begin{enumerate}[a)]
  \item $\mathbb{D} = \{x,y \in \rm I\!R^+ \}$ for when $x=1$
  \item $\mathbb{D} = \{x,y \in \mathbb{Z} \} $
  \item $\mathbb{D} = \{x,y \in \rm{ I\!R} |(x,y) \neq 0 \}$
\end{enumerate}
\end{solution}

\begin{problem} Section 1.6, Exercise 6, page 78. 
\end{problem}
\begin{solution}
\end{solution}

\begin{problem} Section 1.6, Exercise 14 d), page 79.  
\end{problem}
\begin{solution}\ \\
There is someone in this class who has been to France. \\
Everyone who goes to France visits the Louvre. \\
Therefore, someone in this class has visited the Louvre \\ \ \\
\begin{align*}
  P(x):&\; x \text{ is in this class} \\
  Q(x):&\; x \text{ has been to France} \\
  R(x):&\; x \text{ has visited the Lourve}
\end{align*}
\begin{align}
  \exists x(P(x) \land Q(x)& &\textbf{Given} \\
  \forall x(Q(x) \rightarrow R(x))& &\textbf{Given} \\
  P(a) \land Q(a)& &\textbf{Exisitential instantiation on (1)} \\
  Q(a) \rightarrow R(a)& &\textbf{Universal instatiation on (2)} \\
  P(a)& &\textbf{Simplification on (3)} \\
  Q(a)& &\textbf{Simplification on (3)} \\
  R(a)& &\textbf{Modus Ponnes on (4) and (6)} \\
  P(a) \land R(a)& &\textbf{Conjunction on (5) and (7)} \\
  \exists x(P(x) \land R(x)) \qed& &\textbf{Existential generation on (8)}
\end{align}
\end{solution}

\begin{problem} Section 1.7, Exercise 18, page 91.
\end{problem}
\begin{solution} Prove that if n is an integer and 3n + 2 is even, then n is even using
\begin{enumerate}
  \item a proof by contraposition.
  
  \item a proof by contradiction
\end{enumerate}
\end{solution}

\begin{problem} Section 1.7, Exercise 22, page 91.  \textsl{Prove by contradiction.}
\end{problem}
\begin{solution}
\end{solution}

\begin{problem} Section 1.7, Exercise 24, page 91.  \textsl{Prove by contradiction.}
\end{problem}
\begin{solution}
\end{solution}

\begin{problem} Let $n>1$ be an integer. \textsl{Prove by contradiction} that 
if $n$ is a perfect square, then $n+3$ cannot be a perfect square. 
\end{problem}
\begin{solution}
\end{solution}

\goodbreak
\checklist
\end{document}

\documentclass{article}
\usepackage{amsmath,amssymb,amsthm,latexsym,paralist}

\theoremstyle{definition}
\newtheorem{problem}{Problem}
\newtheorem*{solution}{Solution}
\newtheorem*{resources}{Resources}

\newcommand{\name}[2]{\noindent\textbf{Name: #1}\hfill \textbf{Section: #2}}
\newcommand{\honor}{\noindent On my honor, as an Aggie, I have neither
  given nor received any unauthorized aid on any portion of the
  academic work included in this assignment. Furthermore, I have
  disclosed all resources (people, books, web sites, etc.) that have
  been used to prepare this homework. \\[2ex]
 \textbf{Signature:} \underline{\hspace*{10cm}} }
 
\newcommand{\checklist}{\noindent\textbf{Checklist:}
\begin{compactitem}[$\Box$] 
\item Did you type in your name and section? 
\item Did you disclose all resources that you have used? \\
(This includes all people, books, websites, etc.\ that you have consulted.)
\item Did you sign that you followed the Aggie Honor Code? 
\item Did you solve all problems? 
\item Did you submit the .tex and .pdf files of your homework to the correct link on eCampus?
\item Did you submit a signed hardcopy of the pdf file in class? 
\end{compactitem}
}

\newcommand{\problemset}[1]{\begin{center}\textbf{Problem Set #1}\end{center}}
\newcommand{\duedate}[2]{\begin{quote}\textbf{Due dates:} Electronic
    submission of \textsl{yourLastName-yourFirstName-hw2.tex} and 
    \textsl{yourLastName-yourFirstName-hw2.pdf} files of this homework is due on
    \textbf{#1} on \texttt{http://ecampus.tamu.edu}. You will see two separate links
    to turn in the .tex file and the .pdf file separately. Please do not archive or compress the files.  
    A signed paper copy of the pdf file is due on \textbf{#2} \textsl{at the beginning of class}.
    \textbf{If any of the three submissions are missing, your work will not be graded.}\end{quote} }

\newcommand{\N}{\mathbf{N}}
\newcommand{\R}{\mathbf{R}}
\newcommand{\Z}{\mathbf{Z}}


\begin{document}
\vspace*{-15mm}
\begin{center}
{\large
CSCE 222 [Sections 502, 503] Discrete Structures for Computing\\[.5ex]
Spring 2017 -- Hyunyoung Lee\\}
\end{center}
\problemset{2}
\duedate{Friday, 2/3/2017 before 11:00 a.m.}{Friday, 2/3/2017}
\name{Joseph Martinsen}{503}
\begin{resources} (All people, books, articles, web pages, etc.\ that
  have been consulted when producing your answers to this homework.)
\end{resources}
\honor

\smallskip

\begin{problem} Section 1.4, Exercise 30, page 54. 
\end{problem}
\begin{solution}
\begin{enumerate}[a)]
  \ \\ $x \in \{1,2,3 \} \quad y \in \{1,2,3\}$
  \item
  \begin{align*}
    \exists x \; P(x,3) \equiv& P(1,3) \lor P(2,3) \lor P(3,3)
  \end{align*}
  
  \item
  \begin{align*}
    \forall y \; P(1,y) \equiv& P(1,1) \land P(1,2) \land P(1,3)
  \end{align*}
  \item
  \begin{align*}
    \exists y \; \neg P(2,y) \equiv& \neg P(2,1) \lor \neg P(2,2) \lor \neg P(2,3)
  \end{align*}
  \item
  \begin{align*}
    \forall x \; \neg P(x,2) \equiv \neg P(1,2) \land \neg P(2,2) \land \neg P(3,2)
  \end{align*}
\end{enumerate}
\end{solution}

\begin{problem} Section 1.4, Exercise 36, page 55.
\end{problem}
\begin{solution}
\begin{enumerate}[a)] \ \\
  \item When $x=1, \; 1^2 = 1$ and $x=0, \; 0^2 = 0$ \\
        $\therefore$ The counter examples are $x=0,1$
  \item When $x=\sqrt{2}, \; (\sqrt{2})^2 = 2$ and $x=-\sqrt{2}, \; (-\sqrt{2})^2 = 2$ \\
        $\therefore$ The counter examples are $x=\pm \sqrt{2}$
  \item When $x=0, \; |0| = 0$ it follows that $0 \ngtr 0$ \\
        $\therefore$ The counter example is $x=0$
\end{enumerate}
\end{solution}

\begin{problem} Section 1.5, Exercise 28 b), c), e), and i), page 67. 
\textsl{Justify your answer or give a counterexample.}
\end{problem}
\begin{solution} \ \\
b) $\forall x \exists y (x = y^2) $ is \textbf{false} for when $x = -1$ because there is no real number that satisfies $y^2 = -1$ \vspace{8pt} \\
c) $\exists x \forall y(xy = 0)$ is \textbf{true} for all $y$ when $ x=0$ \vspace{8pt}\\
e) $\forall x (x \neq 0 \rightarrow \exists y (xy = 1))$ for all $x$ that does not equal 0 there exists a real number $y$ that is equivalent to $\dfrac{1}{x}$ and it can be shown that $x \cdot \dfrac{1}{x} = 1 \therefore$ the statement is \textbf{true} \vspace{8pt}\\
i) $\forall x \exists y (x + y = 2 \land 2x - y = 1)$ \\
For when $x = 0$ the equation $x+y=2$ the solution for $y$ is 2. Also for when $x = 0$ the equation $2x-y=1$ the solution for $y$ is -1 \\
$\therefore$ the statement is \textbf{false} with $x=0$ as a counterexample.
\end{solution}

\begin{problem} Section 1.5, Exercise 46, page 68.
\textsl{Justify your answer or give a counterexample.}
\end{problem}
\begin{solution}\ \\ $\exists x \forall y \; (x \leq y^2)$
\begin{enumerate}[a)]
  \item $\mathbb{D} = \{x,y \in \rm I\!R^+ \}$ for when $x=1$
  \item $\mathbb{D} = \{x,y \in \mathbb{Z} \} $
  \item $\mathbb{D} = \{x,y \in \rm{ I\!R} |(x,y) \neq 0 \}$
\end{enumerate}
\end{solution}

\begin{problem} Section 1.6, Exercise 6, page 78. 
\end{problem}
\begin{solution}
\begin{align*}
  P:& \text{ It rains} \\
  Q:& \text{ It is foggy} \\
  R:& \text{ Sailing race will be held} \\
  S:& \text{ Lifesaving demonstration occurs} \\
  T:& \text{ Trophy is awarded}
\end{align*}
\setcounter{equation}{0}
\begin{align}
  (\neg P \lor \neg Q) \rightarrow (R \land S)& &\textbf{Given} \\
  R \rightarrow T& & \textbf{Given} \\
  \neg T& &\textbf{Given} \\
  \neg R& &\textbf{Modus Tollens on (2) and (3)} \\
  \neg(R \land S) \rightarrow \neg(\neg P \lor \neg Q)& &\textbf{Contrapositive of (1)} \\
  (\neg R \lor \neg S) \rightarrow (P \land Q)& &\textbf{By DeMorgan's (5)} \\
  \neg R \lor \neg S& &\textbf{Addition of (4) and }\neg S \\
  P \land Q& &\textbf{By Modus Ponnes on (6) and (7)} \\
  P& &\textbf{Simplification on (8)}
\end{align}
$\therefore$ It will rain and there will be no race :(
\end{solution}

\begin{problem} Section 1.6, Exercise 14 d), page 79.  
\end{problem}
\begin{solution}\ \\
There is someone in this class who has been to France. \\
Everyone who goes to France visits the Louvre. \\
Therefore, someone in this class has visited the Louvre \\ \ \\
\begin{align*}
  P(x):&\; x \text{ is in this class} \\
  Q(x):&\; x \text{ has been to France} \\
  R(x):&\; x \text{ has visited the Lourve}
\end{align*}
\setcounter{equation}{0}
\begin{align}
  \exists x(P(x) \land Q(x)& &\textbf{Given} \\
  \forall x(Q(x) \rightarrow R(x))& &\textbf{Given} \\
  P(a) \land Q(a)& &\textbf{Exisitential instantiation on (1)} \\
  Q(a) \rightarrow R(a)& &\textbf{Universal instatiation on (2)} \\
  P(a)& &\textbf{Simplification on (3)} \\
  Q(a)& &\textbf{Simplification on (3)} \\
  R(a)& &\textbf{Modus Ponnes on (4) and (6)} \\
  P(a) \land R(a)& &\textbf{Conjunction on (5) and (7)} \\
  \exists x(P(x) \land R(x)) \qed& &\textbf{Existential generation on (8)}
\end{align}
\end{solution}

\begin{problem} Section 1.7, Exercise 18, page 91.
\end{problem}
\begin{solution} \ \\
\begin{align*}
  P:&\; 3n + 2 \text{ is even} \\
  Q:& \; n \text{ is even} \\
  P &\rightarrow Q
\end{align*}
If $n \in \mathbb{Z}$ and  3n + 2 is even, $n$ is even
\begin{enumerate}
  \item a proof by contraposition $\lnot Q \rightarrow \lnot P$ \\ \ \\
  If n is odd, then 3n + 2 is odd \\
  Since n is assumed to be odd, it must be of the form $n = 2k + 1$ where $k \in \mathbb{Z}$
  \begin{align*}
    3n + 2 \\
    3(2k + 1) + 2 \\
    6k + 3 + 2 \\
    6k + 4 + 1 \\
    2(3k + 2) + 1 \\
    2m + 1& \quad \textbf{where $m = 3k + 2$}
  \end{align*}
  2m + 1 is odd so it follows that If $n \in \mathbb{Z}$ and  3n + 2 is even, $n$ is even by contraposition

  \item a proof by contradiction $P \land \lnot Q$ \\ \ \\
  Assume 3n + 2 is even but n is odd. Since n is odd it must be in the form of n = 2k + 1 where $k \in \mathbb{Z}$

    \begin{align*}
    3n + 2 \\
    3(2k + 1) + 2 \\
    6k + 3 + 2 \\
    6k + 4 + 1 \\
    2(3k + 2) + 1 \\
    2m + 1& \quad \textbf{where $m = 3k + 2$}
  \end{align*}
  2m + 1 is odd which contradicts the assumption that 3n + 2 is even so it follows that If $n \in \mathbb{Z}$ and  3n + 2 is even, $n$ is even by contradiction  
\end{enumerate}
\end{solution}

\begin{problem} Section 1.7, Exercise 22, page 91.  \textsl{Prove by contradiction.}
\end{problem}
\begin{solution}
Assume that if you pick three socks from a drawer containing just blue socks and black socks, that you will not get a pair of blue socks or a pair of black socks. Here the domain of socks within the drawer are \{blue, black\}. For the first two picks, the socks must be of the order [blue, black] or [black, blue]. If it was any other case, a pair of socks would be obtained. For the third pick, there are two options. One option occurs when the third sock is black, which results in [blue, black,black] or [black, blue, black]. The second option occurs when the thrid sock is blue, which results in [blue, black, blue] or [black, blue, blue]. Both of these options result in a pair of socks occuring which contradicts the premise that no pairs would occur \\
$\therefore$ by proof by contradiction, you will have a pair of blue or a pair of black socks when picking from this drawer. 
\end{solution}

\begin{problem} Section 1.7, Exercise 24, page 91.  \textsl{Prove by contradiction.}
\end{problem}
\begin{solution}
\end{solution}

\begin{problem} Let $n>1$ be an integer. \textsl{Prove by contradiction} that 
if $n$ is a perfect square, then $n+3$ cannot be a perfect square. 
\end{problem}
\begin{solution}
\end{solution}

\goodbreak
\checklist
\end{document}

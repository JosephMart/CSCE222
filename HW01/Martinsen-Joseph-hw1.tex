\documentclass{article}
\usepackage{amsmath,amssymb,amsthm,latexsym,paralist}

\theoremstyle{definition}
\newtheorem{problem}{Problem}
\newtheorem*{solution}{Solution}
\newtheorem*{resources}{Resources}

\newcommand{\name}[2]{\noindent\textbf{Name: #1}\hfill \textbf{Section: #2}}
\newcommand{\honor}{\noindent On my honor, as an Aggie, I have neither
  given nor received any unauthorized aid on any portion of the
  academic work included in this assignment. Furthermore, I have
  disclosed all resources (people, books, web sites, etc.) that have
  been used to prepare this homework. \\[2ex]
 \textbf{Signature:} \underline{\hspace*{10cm}} }
 
\newcommand{\checklist}{\noindent\textbf{Checklist:}
\begin{compactitem}[$\Box$] 
\item Did you type in your name and section? 
\item Did you disclose all resources that you have used? \\
(This includes all people, books, websites, etc.\ that you have consulted.)
\item Did you sign that you followed the Aggie Honor Code? 
\item Did you solve all problems? 
\item Did you submit the .tex and .pdf files of your homework to the correct link on eCampus?
\item Did you submit a signed hardcopy of the pdf file in class? 
\end{compactitem}
}

\newcommand{\problemset}[1]{\begin{center}\textbf{Problem Set #1}\end{center}}
\newcommand{\duedate}[2]{\begin{quote}\textbf{Due dates:} Electronic
    submission of \textsl{yourLastName-yourFirstName-hw1.tex} and 
    \textsl{yourLastName-yourFirstName-hw1.pdf} files of this homework is due on
    \textbf{#1} on \texttt{http://ecampus.tamu.edu}. You will see two separate links
    to turn in the .tex file and the .pdf file separately. Please do not archive or compress the files.  
    A signed paper copy of the pdf file is due on \textbf{#2} \textsl{at the beginning of class}.
    \textbf{If any of the three submissions are missing, your work will not be graded.}\end{quote} }

\newcommand{\N}{\mathbf{N}}
\newcommand{\R}{\mathbf{R}}
\newcommand{\Z}{\mathbf{Z}}


\begin{document}
\vspace*{-15mm}
\begin{center}
{\large
CSCE 222 [Sections 502, 503] Discrete Structures for Computing\\[.5ex]
Spring 2017 -- Hyunyoung Lee\\}
\end{center}
\problemset{1}
\duedate{Friday, 1/27/2017 before 11:00 a.m.}{Friday, 1/27/2017}
\name{Joseph Martinsen}{503}
\begin{resources}
  The Textbook \textit{Discrete Mathematics and Its Applications} \\
  Also, familiar with the topics because I am retaking the course...
\end{resources}
\honor

\smallskip

\begin{problem} Section 1.1, Exercise 6 on page 13.
\end{problem}
\begin{solution} \ \\
  \begin{center}
      \begin{tabular}{|c||c|c|c|}
        \hline
         Smartphone & RAM (MB) & ROM (GB) & PIXELS (MP) \\
         \hline
         A & 256 & 32 & 8 \\
         \hline
         B & 288 & 64 & 4 \\
         \hline
         C & 128 & 32 & 5 \\
         \hline
      \end{tabular}
    \end{center}
    
    \begin{enumerate}[(a)]
      \item \textbf{Phrase:} Smartphone B has the most RAM of these three smart-phones. \\
        \textbf{Logic:} $(B_{RAM} > A_{RAM}) \wedge (B_{RAM} > C_{RAM})$ \\
        \textbf{Analysis:} 
        \begin{align*}
          (288 > 256) & \wedge (288 > 128) \\
          T & \wedge T \\
          &T
        \end{align*}
        $\therefore$ The \textbf{Phrase} is $T$
          
      \item \textbf{Phrase:} Smartphone C has more ROM or a higher resolution camera than Smartphone B. \\
        \textbf{Logic:} $(C_{ROM} > B_{ROM}) \lor (C_{PIX} > B_{PIX})$ \\
        \textbf{Analysis:} 
        \begin{align*}
          (32 > 64) & \lor (5 > 4) \\
          F & \lor T \\
          &T
        \end{align*}
        $\therefore$ The \textbf{Phrase} is $T$
      
      \item \textbf{Phrase:} Smartphone B has more RAM, more ROM, and a higher resolution camera than Smartphone A. \\
        \textbf{Logic:} $(B_{RAM} > A_{RAM}) \land (B_{ROM} > A_{ROM}) \land (B_{PIX} > A_{PIX})$ \\
        \textbf{Analysis:} 
        \begin{align*}
          (288 > 256) \wedge (64 &> 32) \land (4 > 8) \\
          T \land &T \land F \\
          &F
        \end{align*}
        $\therefore$ The \textbf{Phrase} is $F$
      
      \item \textbf{Phrase:} If Smartphone B has more RAM and more ROM than Smartphone C, then it also has a higher resolution camera.\\
        \textbf{Logic:} $((B_{RAM} > C_{RAM}) \land (B_{ROM} > C_{ROM})) \rightarrow (B_{PIX} > C_{PIX})$ \\
        \textbf{Analysis:} 
        \begin{align*}
          ((288 > 128) \wedge (64 > 32)) &\rightarrow (4 > 5) \\
          (T \land T) &\rightarrow F \\
          T &\rightarrow F \\
          &F
        \end{align*}
        $\therefore$ The \textbf{Phrase} is $F$
        
      \item \textbf{Phrase:} Smartphone A has more RAM than Smartphone B if and only if Smartphone B has more RAM than Smart-phone A. \\
        \textbf{Logic:} $(A_{RAM} > B_{RAM}) \leftrightarrow  (B_{RAM} > A_{RAM})$ \\
        \textbf{Analysis:} 
        \begin{align*}
          (256 > 288) &\leftrightarrow (288 > 256) \\
          F &\leftrightarrow T\\
          &F
        \end{align*}
        $\therefore$ The \textbf{Phrase} is $F$
        
    \end{enumerate}
\end{solution}

\begin{problem} Section 1.1, Exercise 16 on page 14.
\end{problem}
\begin{solution} \ \\
\begin{enumerate}[(a)]
  \item $2 + 2 = 4$ if and only if $1 + 1 = 2$ \\
  $T \leftrightarrow T$ \\
  $\therefore T$
  
  \item $1 + 1 = 2$ if and only if $2 + 3 = 4$ \\
  $T \leftrightarrow F$ \\
  $\therefore F$
  
  \item $1 + 1 = 3$ if and only if monkeys can fly. \\
  $F \leftrightarrow F$ \\
  $\therefore T$
  
  \item $ 0 > 1$ if and only if $2 > 1$ \\
  $F \leftrightarrow T$ \\
  $\therefore F$
\end{enumerate}
\end{solution}

\begin{problem} Section 1.1, Exercise 22 a) -- e) on page 14.
\end{problem}
\begin{solution} \ \\
\begin{enumerate}[(a)]
  \item It is necessary to wash the boss’s car to get promoted. \\
  $p:$ one washes the boss's car \\
  $q:$ one gets promoted\\
  \textbf{if $p$, then $q$:} if one washes the boss's car then one gets promoted
  
  \item Winds from the south imply a spring thaw. \\
  $p:$ there are winds from the south \\
  $q:$ there will be a spring thaw \\
  \textbf{if $p$, then $q$:} if there are winds from the south then there will be a spring thaw
  
  \item A sufficient condition for the warranty to be good is that you bought the computer less than a year ago. \\
  $p:$ you bought the computer less than a year ago \\
  $q:$ sufficient condition for the warranty is good \\
  \textbf{if $p$, then $q$:} if you bought the computer less than a year ago then sufficient condition for the warranty is good
  
  \item Willy gets caught whenever he cheats. \\
  $p:$ he cheats \\
  $q:$ gets caught \\
  \textbf{if $p$, then $q$:} if Willy cheats then he gets caught.
  
  \item You can access the website only if you pay a subscription fee. \\
  $p:$ you pay a subscription fee \\
  $q:$ you can access the website \\
  \textbf{if $p$, then $q$:} if you pay a subscription fee then you can access the website

\end{enumerate}
\end{solution}

\begin{problem} Section 1.1, Exercise 40 on page 16.
\end{problem}
\begin{solution} \ \\
\noindent Explain, without using a truth table, why $(p \lor \neg q) \land
(q \lor \neg r) \land (r \lor \neg p)$ is true when $p$, $q$, and $r$ have the
same truth value and it is false otherwise. \\

\noindent\textbf{Case 1: }If p,q, and r are all T, then the results is 
\begin{equation*}
\begin{gathered}
  (T \lor \neg T) \land (T \lor \neg T) \land (T \lor \neg T) \\
  T \land T \land T\\
  T
\end{gathered}
\end{equation*}

\noindent\textbf{Case 2: }If p, q, and r are all F, then the result is
\begin{equation*}
\begin{gathered}
  (F \lor \neg F) \land (F \lor \neg F) \land (F \lor \neg F) \\
  T \land T \land T\\
  T
\end{gathered}
\end{equation*}

\noindent\textbf{Case 3: }If p is T and q is F, r will be set to F, then the reulst is
\begin{equation*}
\begin{gathered}
(T \lor \neg F) \land (T \lor \neg F) \land (F \lor \neg T) \\
T \land T \land F\\
F
\end{gathered}
\end{equation*}

Regardless of r, the result is the same.

\noindent\textbf{Case 4: }If p is F and q is T, r will be set to F, then the reulst is
\begin{equation*}
\begin{gathered}
(F \lor \neg T) \land (T \lor \neg F) \land (F \lor \neg F) \\
F \land T \land T\\
F
\end{gathered}
\end{equation*}
\noindent Regardless of r, the result is the same. \\
\noindent$\therefore$ If p, q, and r are not the same truth value, the proposition will result in false.  
\end{solution}

\begin{problem} Section 1.1, Exercise 42 on page 16. \textsl{Explain.}
\end{problem}
\begin{solution}
\begin{enumerate}[(a)] $x := 1$
  \item if $x + 2 = 3$ \textbf{then} $x := x + 1$ \\
  Since $x$ is 1 to begin with, $1 + 2 = 3$ is true \\
  $\therefore$ $x := x + 1$ will occur and $x$ will result in $x = 2$
  
  \item if $(x + 1 = 3)$ OR $(2x + 2 = 3)$ \textbf{then} $x := x + 1$ \\
  Since $x$ is 1 to begin with, $(1 + 1 = 3)$ is false and $(2(1) + 2 = 3)$ is false \\
  $\therefore$ the statement will not enter and $x$ will stay 1
  
  \item if $(2x + 3 = 5)$ AND $(3x + 4 = 7)$ \textbf{then} $x := x + 1$ \\
  Since $x$ is 1 to begin with, $(2(1) + 3 = 5)$ is true and $(3(1) + 4 = 7)$ is true \\
  $\therefore$ $x := x + 1$ will execute and the result of $x$ is 2  
  
  \item if $(x + 1 = 2)$ XOR $(x + 2 = 3)$ \textbf{then} $x := x + 1$ \\
  Since $x$ is 1 to begin with, $(1 + 1 = 2)$ is true and $(1 + 2 = 3)$ is true \\
  $\therefore$ the statement $x := x+1$ will not execute and $x$ will stay as 1
  
  \item if $x < 2$ then $x := x + 1$ \\
  Since $x$ is 1 to begin with, $1 < 2$ is true \\
  $\therefore$ the statement will execute and $x$ will result in being 2 \\
\end{enumerate}
\end{solution}

\begin{problem} Section 1.2, Exercise 6 on page 22.
\end{problem}
\begin{solution}
  \begin{align*}
  u: & \quad \text{You can upgrade your operating system} \\
  b_{32}: & \quad \text{You have a 32-bit processor} \\
  b_{64}: & \quad \text{You have a 64-bit processor} \\
  g_1:    & \quad \text{Your processor runs at 1 GHz or faster} \\
  g_2:    & \quad \text{Your processor runs at 2 GHz or faster} \\
  r_1:    & \quad \text{Your processor has at least 1 GB RAM} \\
  r_2:    & \quad \text{Your processor has at least 2 GB RAM} \\
  h_{16}: & \quad \text{You have at least 16 GB free hard disk space} \\
  h_{32}: & \quad \text{You have at least 32 GB free hard disk space} \\
  u \rightarrow & \quad (b_{32} \land g_1 \land r_1 \land h_{16}) \lor (b_{64} \land g_2 \land r_2 \land h_{32})
  \end{align*}
\end{solution}

\begin{problem} Section 1.2, Exercise 22 on page 23. \textsl{Explain.}
\end{problem}
\begin{solution} \ \\
  Since both $A$ and $B$ both said that they are knights the following conclutions can be drawn: \\
  $A$ is either a knight or a knave and $B$ is knight or a knave
\end{solution}

\begin{problem} Section 1.3, Exercise 10 b) and c), page 35.
\end{problem}
\begin{solution}
  \begin{enumerate}[(a)]
  \item
    \item $((p \rightarrow q) \land (q \rightarrow r)) \rightarrow (p \rightarrow r)$
    \begin{center}
      \begin{tabular}{|c c c || c c c c c|}
      \hline
        $p$ & $q$ & $r$ & $p \rightarrow q$ & $q \rightarrow r$ & $p \rightarrow r$ & $(p \rightarrow q) \land (q \rightarrow r)$ & $((p \rightarrow q) \land (q \rightarrow r)) \rightarrow (p \rightarrow r)$ \\
        \hline
        F & F & F & T & T & T & T & T \\
        F & F & T & T & F & T & F & T \\
        F & T & F & T & T & T & T & T \\
        F & T & T & T & T & T & T & T \\
        \hline
        T & F & F & F & T & F & F & T \\
        T & F & T & F & T & T & F & T \\
        T & T & F & T & F & F & F & T \\
        T & T & T & T & T & T & T & T \\
        \hline
      \end{tabular}
    \end{center}
    This is the definition of hypothetical syllogism
    
    \item $((p \land (p \rightarrow q)) \rightarrow q$
    \begin{center}
      \begin{tabular}{|c c|| c c c|}
      \hline
        $p$ & $q$ & $p \rightarrow q$ & $p \land (p \rightarrow q)$ & $((p \land (p \rightarrow q)) \rightarrow q$\\
        \hline
        F & F & T & F & T \\
        F & T & T & F & T \\
        T & F & F & F & T \\
        T & T & T & T & T \\
        \hline
      \end{tabular}
    \end{center}
    This is the definition of modus ponens    
  \end{enumerate}
\end{solution}

\noindent
Solve the following two problems by developing a series of logical equivalences, 
as shown in class and also in Examples~7 and~8 on page 30.

\begin{problem} Show that $(p\rightarrow q) \lor (p\rightarrow r)$ and 
$p\rightarrow (q\lor r)$ are logically equivalent. 
\end{problem}
\begin{solution}
  \begin{align*}
    (p\rightarrow q) \lor (p\rightarrow r) &\equiv (\neg p \lor q) \lor (\neg p \lor r) &\textbf{By $A \rightarrow B \equiv \neg A \lor B$} \\
  &\equiv \neg p \lor \neg p \lor q \lor r &\textbf{By Associative Laws} \\
      &\equiv \neg p \lor q \lor r & \textbf{By Simplification} \\
      &\equiv \neg p \lor (q \lor r) &\textbf{By Associative Laws} \\
      &\equiv p \rightarrow (q \lor r) \qed &\textbf{By $\neg A \lor B \equiv A \rightarrow B$}
  \end{align*}
\end{solution}

\begin{problem} Show that $(p\lor q)\land (\neg p\lor r)\rightarrow (q\lor r)$
is a tautology.
\end{problem}
\begin{solution}
  \begin{align*}
    (p\lor q)\land (\neg p\lor r)\rightarrow (q\lor r) &\equiv
    \neg((p\lor q)\land (\neg p\lor r))\lor (q\lor r) &
    \textbf{By $A \rightarrow B \equiv \neg A \lor B$} \\
    &\equiv (\neg(p \lor q) \lor \neg(\neg p \lor r)) \lor (q\lor r) & \textbf{By DeMorgan's} \\
    &\equiv (\neg p \land \neg q) \lor (p \land \neg r) \lor (q \lor r) & \textbf{By DeMorgan's} \\
    &\equiv (\neg p \land \neg q) \lor q \lor (p \land \neg r) \lor r & \textbf{By Associative Laws} \\
    & \equiv ((\neg p \lor q) \land (\neg q \lor q)) \lor (p \land \neg r) \lor r & \textbf{By Distributive Laws} \\
    & \equiv ((\neg p \lor q) \land T) \lor (p \land \neg r) \lor r & \textbf{By $\neg A \lor A \equiv T$} \\
    & \equiv \neg p \lor q \lor (p \land \neg r) \lor r & \textbf{By $A \land T \equiv A$} \\
    & \equiv \neg p \lor q \lor ((p \lor r) \land ( r \lor \neg r)) & \textbf{By Distributive Laws} \\
    & \equiv \neg p \lor q \lor ((p \lor r) \land T) & \textbf{By $\neg A \lor A \equiv T$} \\
    & \equiv \neg p \lor q \lor p \lor r & \textbf{By $A \land T \equiv A$} \\
    &\equiv T \lor q \lor r & \textbf{By $\neg A \lor A \equiv T$} \\
    &\equiv T \lor r & \textbf{By $T \lor A \equiv T$} \\
    &\equiv T \qed & \textbf{By $T \lor A \equiv T$}
  \end{align*}
\end{solution}

\goodbreak
\checklist
\end{document}

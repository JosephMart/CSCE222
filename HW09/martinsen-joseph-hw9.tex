\documentclass{article}
\usepackage{amsmath,amssymb,amsthm,latexsym,paralist}
\usepackage{enumitem}
\input xypic
\theoremstyle{definition}
\newtheorem{problem}{Problem}
\newtheorem*{solution}{Solution}
\newtheorem*{resources}{Resources}

\newcommand{\name}[2]{\noindent\textbf{Name: #1}\hfill \textbf{Section: #2}}
\newcommand{\honor}{\noindent On my honor, as an Aggie, I have neither
  given nor received any unauthorized aid on any portion of the
  academic work included in this assignment. Furthermore, I have
  disclosed all resources (people, books, web sites, etc.) that have
  been used to prepare this homework. \\[2ex]
 \textbf{Signature:} \underline{\hspace*{8cm}} }
 
\newcommand{\checklist}{\noindent\textbf{Checklist:}
\begin{compactitem}[$\Box$] 
\item Did you type in your name and section? 
\item Did you disclose all resources that you have used? \\
(This includes all people, books, websites, etc.\ that you have consulted)
\item Did you sign that you followed the Aggie honor code? 
\item Did you solve all problems? 
\item Did you submit both the .tex and .pdf files of your homework separately 
to the correct link on eCampus?
\item Did you submit a signed hardcopy of the pdf file in class? 
\end{compactitem}
}

\newcommand{\problemset}[1]{\begin{center}\textbf{Problem Set #1}\end{center}}
\newcommand{\duedate}[2]{\begin{quote}\textbf{Due dates:} Electronic
    submission of \textsl{yourLastName-yourFirstName-hw9.tex} and 
    \textsl{yourLastName-yourFirstName-hw9.pdf} files of this homework is due on
    \textbf{#1} on \texttt{http://ecampus.tamu.edu}. You will see two separate links
    to turn in the .tex file and the .pdf file separately. Please do not archive or compress the files.  
    A signed paper copy of the pdf file is due on \textbf{#2} at the beginning of class.
    \textbf{If any of the three submissions are missing, your work will not be graded.}
    \textbf{Late submissions of any form will not be accepted.}\end{quote} }

\newcommand{\N}{\mathbf{N}}
\newcommand{\R}{\mathbf{R}}
\newcommand{\Z}{\mathbf{Z}}


\begin{document}
\vspace*{-18mm}
\begin{center}
{\large
CSCE 222 [Sections 502, 503] Discrete Structures for Computing\\[.5ex]
Spring 2017 -- Hyunyoung Lee\\}
\end{center}
\problemset{9}
\duedate{Wednesday, 4/19/2017 \textit{before} the beginning of class}{Wednesday, 4/19/2017}
\name{ Joseph Martinsen }{503}
\begin{resources} (All people, books, articles, web pages, etc.\ that
  have been consulted when producing your answers to this homework) \\
  http://mathworld.wolfram.com/EquivalenceRelation.html
\end{resources}
\honor

\begin{problem} (16 points)
Section 9.1, Exercise 6, page 581
\end{problem}
\begin{solution} \ \\
  \begin{enumerate}[label=(\alph*)]

    \item $$ R: x+y = 0$$ \\
  
      \textbf{Reflexive} if $(a,a) \in R$ for every element $a \in A$ \\
      For $x = y = 1$, $1 + 1 =2 \neq 0$ \\
      $\therefore R$ is not reflexive
    
      \textbf{Symmetric} if $(b,a) \in R$ whenever $(a,b) \in R$ for all $a,b \in A$ \\
      For $x,y \in \mathbb{R}$, $ x+ y = 0 = y + x$ \\
      $\therefore R$ is symmetric
    
      \textbf{Anit-Symmetric} if $(a,b) \in R$ and $(b,a) \in R$, then $a=b$ for all $a,b \in A$ \\
      For $x=1$, $y=-1$, it follows $1 + (-1) = 0$ and $-1 + 1 = 0$ eventhough $x \neq y$\\
      $\therefore R$ is not anti-symmetric 
    
      \textbf{Transitive} if whenever $(a,b) \in R$ and $(b,a) \in R$ then $(a,c) \in R$, for all $a,b,c \in A$ \\
      For 1 and $-1$, $R(-1,1): -1 + 1 = 0$ and $R(1,-1): 1 + (-1) = 0$ yet $R(1,1): 1 + 1 \neq 0$ \\
      $\therefore R$ is not transitive

    \item $$ R: x = \pm y $$ \\
  
      \textbf{Reflexive} if $(a,a) \in R$ for every element $a \in A$ \\
      For $x = y = x$, $x = x$ \\
      $\therefore R$ is reflexive
    
      \textbf{Symmetric} if $(b,a) \in R$ whenever $(a,b) \in R$ for all $a,b \in A$ \\
      $x = \pm y$ \\
      $y = \pm x$ \\
      $\therefore R$ is symmetric
    
      \textbf{Anit-Symmetric} if $(a,b) \in R$ and $(b,a) \in R$, then $a=b$ for all $a,b \in A$ \\
      For $-1 = \pm 1$ and $1 = \pm (-1)$ yet $-1 \neq 1$\\
      $\therefore R$ is not anti-symmetric 
    
      \textbf{Transitive} if whenever $(a,b) \in R$ and $(b,a) \in R$ then $(a,c) \in R$, for all $a,b,c \in A$ \\
      For $x = \pm y$ and $y = \pm z$, it follows that $x = \pm z$ \\
      $\therefore R$ is transitive

    \item \begin{align*}
            R: x - y  \quad &\text{is rational}
          \end{align*} 
  
      \textbf{Reflexive} if $(a,a) \in R$ for every element $a \in A$ \\
      For $x = y = x$, $x - x = 0 $, it follows that $ x = x$ and since $x$ is rational, the difference must be rational \\
      $\therefore R$ is reflexive
    
      \textbf{Symmetric} if $(b,a) \in R$ whenever $(a,b) \in R$ for all $a,b \in A$ \\
      The difference between two rationals results in a rational number. It follows $x-y$ is rational and $y-x$ is rational. \\
      $\therefore R$ is symmetric
    
      \textbf{Anit-Symmetric} if $(a,b) \in R$ and $(b,a) \in R$, then $a=b$ for all $a,b \in A$ \\
      For $R(1,0)$ and $R(0,1)$, it follows $1 - 0 = 1$ is rational and $0 - 1 = -1$ is rational yet $0 \neq 1$\\
      $\therefore R$ is not anti-symmetric 
    
      \textbf{Transitive} if whenever $(a,b) \in R$ and $(b,a) \in R$ then $(a,c) \in R$, for all $a,b,c \in A$ \\
      For $R(0,1)$ $R(1,2)$, it follows $0 - 1 = -1$ is rational and $1 - 2 = -1$ is rational yet $0 \neq 2$
      $\therefore R$ is not transitive

    \item $$ R: x = 2y $$ \\
  
      \textbf{Reflexive} if $(a,a) \in R$ for every element $a \in A$ \\
      For $x = y = 1$, $1  \neq 2 \cdot 1$ \\
      $\therefore R$ is not reflexive
    
      \textbf{Symmetric} if $(b,a) \in R$ whenever $(a,b) \in R$ for all $a,b \in A$ \\
      For $x,y \in \mathbb{R}$, $ x = 2y$ and $ y = 2x$, it follows that $ \dfrac{1}{2} y \neq 2y$\\
      $\therefore R$ is not symmetric
    
      \textbf{Anit-Symmetric} if $(a,b) \in R$ and $(b,a) \in R$, then $a=b$ for all $a,b \in A$ \\
      For $x,y \in \mathbb{R}$, $ x = 2y$ and $ y = 2x$, it follows that $ \dfrac{1}{2} y \neq 2y$\\
      $\therefore R$ is not anti-symmetric 
    
      \textbf{Transitive} if whenever $(a,b) \in R$ and $(b,c) \in R$ then $(a,c) \in R$, for all $a,b,c \in A$ \\
      $R(4,2): 4 = 2 \cdot 2 $ and $R(2,1): 2 = 2 \cdot 1$ yet $ 4 \neq 1$ \\
      $\therefore R$ is not transitive

    \item $$ R: xy \geq 0$$ \\
  
      \textbf{Reflexive} if $(a,a) \in R$ for every element $a \in A$ \\
      For $x = y = x$, $x^2 \geq 0$ \\
      $\therefore R$ is reflexive
    
      \textbf{Symmetric} if $(b,a) \in R$ whenever $(a,b) \in R$ for all $a,b \in A$ \\
      For $xy \geq 0 $, $ yx \geq 0$ \\
      $\therefore R$ is symmetric
    
      \textbf{Anit-Symmetric} if $(a,b) \in R$ and $(b,a) \in R$, then $a=b$ for all $a,b \in A$ \\
      For $x=1$, $y=2$, it follows $1 \cdot 2 \geq 0$ and $1\cdot 2 \geq 0$ eventhough $1 \neq 2$\\
      $\therefore R$ is not anti-symmetric 
    
      \textbf{Transitive} if whenever $(a,b) \in R$ and $(b,a) \in R$ then $(a,c) \in R$, for all $a,b,c \in A$ \\
      $R(2,1): 2 \cdot 1 \geq 0$ and $R(2,3): 2 \cdot 3 \geq 0$ yet $2 \neq 3$ \\
      $\therefore R$ is not transitive

    \item $$ R: xy = 0$$ \\
  
      \textbf{Reflexive} if $(a,a) \in R$ for every element $a \in A$ \\
      For $x = y = 1$, $1 \cdot 1 \neq 0$ \\
      $\therefore R$ is not reflexive
    
      \textbf{Symmetric} if $(b,a) \in R$ whenever $(a,b) \in R$ for all $a,b \in A$ \\
      For $x,y \in \mathbb{R}$, $ xy = 0$ it must follow that $ yx = 0$ \\
      $\therefore R$ is symmetric
    
      \textbf{Anit-Symmetric} if $(a,b) \in R$ and $(b,a) \in R$, then $a=b$ for all $a,b \in A$ \\
      For $x=1$, $y=0$, it follows $1 \cdot 0 = 0$ and $0 \cdot 1 = 0$ eventhough $1 \neq 0$\\
      $\therefore R$ is not anti-symmetric 
    
      \textbf{Transitive} if whenever $(a,b) \in R$ and $(b,a) \in R$ then $(a,c) \in R$, for all $a,b,c \in A$ \\
      $R(1,0): 1 \cdot 0 = 0$ and $R(0,3): 0 \cdot 3 = 0$ yet $3 \neq 1$ \\
      $\therefore R$ is not transitive

    \item $$ R: x = 1$$ \\
  
      \textbf{Reflexive} if $(a,a) \in R$ for every element $a \in A$ \\
      For $x = 2$, $2 \neq 1$ \\
      $\therefore R$ is not reflexive
    
      \textbf{Symmetric} if $(b,a) \in R$ whenever $(a,b) \in R$ for all $a,b \in A$ \\
      For $x,y \in \mathbb{R}$, $ x+ y = 0 = y + x$ \\
      $\therefore R$ is symmetric
    
      \textbf{Anit-Symmetric} if $(a,b) \in R$ and $(b,a) \in R$, then $a=b$ for all $a,b \in A$ \\
      For $x=1$, $y=1$ from $R$, it follows $x = y$\\
      $\therefore R$ is anti-symmetric 
    
      \textbf{Transitive} if whenever $(a,b) \in R$ and $(b,a) \in R$ then $(a,c) \in R$, for all $a,b,c \in A$ \\
      Since for R(1,1) is the only case that will hold true, $x = z$ because $ 1 = 1$
      $\therefore R$ is transitive

    \item $$ R: x = 1 \text{ or } y = 1$$ \\
  
      \textbf{Reflexive} if $(a,a) \in R$ for every element $a \in A$ \\
      For $x = y = 2$, $2 \neq 1$ \\
      $\therefore R$ is not reflexive
    
      \textbf{Symmetric} if $(b,a) \in R$ whenever $(a,b) \in R$ for all $a,b \in A$ \\
       $x = 1 \text{ or } y = 1$ implies that $x=1$ and $y=1$ 
      $\therefore R$ is symmetric
    
      \textbf{Anit-Symmetric} if $(a,b) \in R$ and $(b,a) \in R$, then $a=b$ for all $a,b \in A$ \\
      For $R(1,2)$ and $R(1,1)$ holds yet $1 \neq 2$
      $\therefore R$ is not anti-symmetric 
    
      \textbf{Transitive} if whenever $(a,b) \in R$ and $(b,a) \in R$ then $(a,c) \in R$, for all $a,b,c \in A$ \\
      For $R(1,2)$ and $R(1,3)$ holds yet $R(2,3)$ does not hold
      $\therefore R$ is not transitive

  \end{enumerate}
\end{solution}

\begin{problem} (10 points)
We define on the set $\N_1=\{1,2,3,\cdots\}$ of positive integers a
relation $\sim$ such that two positive integers $x$ and $y$ satisfy
$x\sim y$ if and only if $x/y=2^k$ for some integer $k$. 
Show that $\sim$ is an equivalence relation.
\end{problem}
\begin{solution} 
\end{solution} \ \\
      \textbf{Reflexive} if $(a,a) \in R$ for every element $a \in A$ \\
      For $x = y = x$, $x \sim x \rightarrow \dfrac{x}{x} = 1 = 2^0$ \\
      $\therefore R$ is reflexive
      \textbf{Symmetric} if $(b,a) \in R$ whenever $(a,b) \in R$ for all $a,b \in A$ \\
       $R(x,y): x/y = 2^{\log x/y}$ where $k = \log x/y$ since $x/y$ is always a number greater than 0 \\
       $R(y,x): y/x = 2^{\log y/x}$ where $k = \log y/x$ \\
      $\therefore R$ is symmetric
      \textbf{Transitive} if whenever $(a,b) \in R$ and $(b,a) \in R$ then $(a,c) \in R$, for all $a,b,c \in A$ \\
      For $R(x,y)$ and $R(y,z)$ it follows that $R(x,z)$ holds where $k = \log x/z$
      $\therefore R$ is transitive \\
      Since the relation is transitive, symmetric, and reflexive, the relation is an equivalence relation.

\begin{problem} (10 points)
Section 9.5, Exercise 2, page 615
\end{problem}
\begin{solution} \ \\
  \begin{enumerate}[label=(\alph*)]
    \item $R$ is an euq. relation
    \item $R$ is an euq. relation
    \item $R$ is reflexive and symmetric but not transitive. $a$ and $b$ may share a common parent. $b$ and $c$ may share a common parent but it does not follow that $a$ and $c$ have a common parent.
    \item $R$ is not transitive. $a$ and $b$ may have met. $b$ and $c$ may have met but it does not follow that $a$ and $c$ have met.
    \item $R$ is not transitive. $a$ and $b$ may share a common language. $b$ and $c$ may share a common language but it does not follow that $a$ and $c$ share a common language.
  \end{enumerate}
\end{solution}

\begin{problem} (10 points)
Section 9.5, Exercise 16, page 615
\end{problem}
\begin{solution} 
  $$ ((a, b), (c, d) ) \in R \text{ iff } ad = bc$$
  $a \cdot b = b \cdot a$ by communicative property of multiplication. It then follows $( (a,b), (b,a))$ $\therefore R$ is reflexive  \\
  $ad = bc$ relates to $R((a,b),(c,d))$ \\
  $bc = ad$ relates to $R((c,d),(a,b))$ \\
  $\therefore R$ is symmetric. \\
  Given $R0: ((a,b),(c,d))$ or $ad = bc$ and $R0: ((c,d),(e,f))$ or $cf=de$ it follows that $((a,b),(e,f))$ holds because of the following:
  \begin{align*}
    ad &= bc \\
    \dfrac{a}{b} 7= \dfrac{c}{d} \\
    cf &= de \\
    \dfrac{c}{d} &= \dfrac{e}{f} \\
    \dfrac{a}{b} &= \dfrac{e}{f} \\
    af &= eb
  \end{align*}
  $\therefore R$ is is transitive \\
  $\therefore R$ is equivalence relation  
\end{solution}

\begin{problem} (10 points)
Section 9.5, Exercise 58, page 618
\end{problem}
\begin{solution} 
\end{solution}

\begin{problem} (10 points)
Section 9.6, Exercise 4, page 630
\end{problem}
\begin{solution} \ \\
\begin{enumerate}[label=(\alph*)]
  \item This is not a poset because it is not reflexive. 
  \item This is not a poset because it is not antisymetric
  \item Is a poset
  \item Is not a poset because it is not antisymetric 
\end{enumerate}
\end{solution}

\begin{problem} (10 points)
Section 9.6, Exercise 16, page 630.
\ \\
You can answer the subproblem c) by either actually drawing the Hasse diagram 
or by clearly writing out the cover relation of the Hasse diagram instead of 
drawing the diagram.
\end{problem}
\begin{solution} \ \\
\begin{enumerate}[label=(\alph*)]
  \item $\{ 1,1\} , \{ 1,2\} , \{ 1,3\} , \{ 1,4\} , \{ 2,1\} , \{ 2,2\}$
  \item $\{ 3,2\} , \{ 3,3\} , \{ 3,4\} , \{ 4,1\} , \{ 4,2\} , \{ 4,3\}, \{ 4,4\}$
  \item
  $$
\diagram
(4,4) \\
(4,3) \uline \\
(4,2) \uline \\
(4,1) \uline \\
(3,4) \uline \\
(3,3) \uline \\
(3,2) \uline \\
(3,1) \uline \\
(2,4) \uline \\
(2,3) \uline \\
(2,2) \uline \\
(2,1) \uline \\
(1,4) \uline \\
(1,3) \uline \\
(1,2) \uline \\
(1,1) \uline \\
\enddiagram
$$

\end{enumerate}
\end{solution}

\begin{problem} ($8+8\times 2 = 24$ points)
Section 9.6, Exercise 34, page 631.
\ \\ 
First, draw the Hasse diagram or clearly write out the cover relation.  This
part is worth eight points.  Each of the eight subproblems is worth two points.
For subproblems c) and d), if it exists, give the number.
\end{problem}
\begin{solution} \ \\
$$
\diagram
27 & &72 & 48 & 60 \\
\uline & & 36\uline & \uline \urline\\
\uline & 18 \urline & & \ulline 12 \uline\\
\uline \urline &    & \ulline 6 \urline & & \ulline 4\\
9 \uline & & &\ulline 2 \urline\\
\enddiagram
$$

\begin{enumerate}[label=(\alph*)]
  \item 27, 48, 60, 72
  \item 2, 9
  \item No greatest
  \item No least
  \item 18, 36, 72
  \item 18
  \item 2, 4, 6, 12
  \item 12
\end{enumerate}
\end{solution}

\goodbreak
\checklist
\end{document}

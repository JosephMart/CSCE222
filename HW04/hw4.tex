\documentclass{article}
\usepackage{amsmath,amssymb,amsthm,latexsym,paralist}
\usepackage{listings}
\theoremstyle{definition}
\newtheorem{problem}{Problem}
\newtheorem*{solution}{Solution}
\newtheorem*{resources}{Resources}

\newcommand{\name}[2]{\noindent\textbf{Name: #1}\hfill \textbf{Section: #2}}
\newcommand{\honor}{\noindent On my honor, as an Aggie, I have neither
  given nor received any unauthorized aid on any portion of the
  academic work included in this assignment. Furthermore, I have
  disclosed all resources (people, books, web sites, etc.) that have
  been used to prepare this homework. \\[2ex]
 \textbf{Signature:} \underline{\hspace*{10cm}} }
 
\newcommand{\checklist}{\noindent\textbf{Checklist:}
\begin{compactitem}[$\Box$] 
\item Did you type in your name and section? 
\item Did you disclose all resources that you have used? \\
(This includes all people, books, websites, etc.\ that you have consulted.)
\item Did you sign that you followed the Aggie Honor Code? 
\item Did you solve all problems? 
\item Did you submit the .tex and .pdf files of your homework to the correct link on eCampus?
\item Did you submit a signed hardcopy of the pdf file in class? 
\end{compactitem}
}

\newcommand{\problemset}[1]{\begin{center}\textbf{Problem Set #1}\end{center}}
\newcommand{\duedate}[2]{\begin{quote}\textbf{Due dates:} Electronic
    submission of \textsl{yourLastName-yourFirstName-hw4.tex} and 
    \textsl{yourLastName-yourFirstName-hw4.pdf} files of this homework is due on
    \textbf{#1} on \texttt{http://ecampus.tamu.edu}. You will see two separate links
    to turn in the .tex file and the .pdf file separately. Please do not archive or compress the files.  
    A signed paper copy of the pdf file is due on \textbf{#2} \textsl{at the beginning of class}.
    \textbf{If any of the three submissions are missing, your work will not be graded.}\end{quote} }

\newcommand{\N}{\mathbf{N}}
\newcommand{\R}{\mathbf{R}}
\newcommand{\Z}{\mathbf{Z}}


\begin{document}
\vspace*{-15mm}
\begin{center}
{\large
CSCE 222 [Sections 502, 503] Discrete Structures for Computing\\[.5ex]
Spring 2017 -- Hyunyoung Lee\\}
\end{center}
\problemset{4}
\duedate{Friday, 2/24/2017 before 11:00 a.m.}{Friday, 2/24/2017}
\name{Joseph Martinsen}{503}
\begin{resources} (All people, books, articles, web pages, etc.\ that
  have been consulted when producing your answers to this homework.) \\
  \begin{verbatim}
  https://www.youtube.com/watch?v=P2qHss2-aSQ
  https://www.youtube.com/watch?annotation_id=annotation_2862598731&feature=iv&
    index=3&list=PLj68PAxAKGowkG1QYgun4DrByPwsyB04h&src_vid=P2qHss2-aSQ&v=DjfYhHSkWqo
  https://www.youtube.com/watch?annotation_id=annotation_4083618415&feature=iv&src_vid=P2qHss2-aSQ&v=
    Vzqaz4MDGvc
    \end{verbatim}
\end{resources}
\honor

\smallskip

\begin{problem} (10 points)
Let $f_1, f_2, f_3, f_4$ be functions from the set $\N$ of natural numbers
to the set $\R$ of real numbers. Suppose that $f_1= O(f_2)$ and
$f_3=O(f_4)$. Use the \textit{definition} of Big Oh \textit{given in class} to prove that 
$$f_1(n) + f_3(n) = O(\max(|f_2(n)|,  |f_4(n)| ) ).$$
\end{problem}
\begin{solution}
By definition of Big Oh, \\
If $f_1 \in O(f_2)$, then there exist a $k_1 \in \mathbb{R}$ and $n_1 \in \mathbb{N}$ such that $f_1(n) \le Af_2(n)$ for all $n>n_1$\\
In the same manner, if $f_3 \in O(f_4)$, then there exist a $k_2 \in \mathbb{R}$ and $n_2 \in \mathbb{N}$ such that $f_1(n) \le Af_2(n)$ for all $n>n_2$ \\
Let $N = \text{max}(n_1,n_2)$. For all $n > N$ it is true that
\begin{align*}
  f_1(n) + f_3(n) &\le k_1 f_2(n) + k_2 f_4(n)
\end{align*}
Let $K = \text{max}(k_1,k_2)$ and $f_5(n) = \text{max}(f_2(n),f_4(n))$
\begin{align*}
  f_1(n) + f_3(n) &\le K[f_2(n) + f_4(n)] \\
  &\le K f_5(n) = K \;\text{max}(f_2(n),f_4(n)) \\
    f_1(n) + f_3(n) &\le K \;\text{max}(f_2(n),f_4(n))
\end{align*}
By definition of Big Oh, for $n > N$
\begin{align*}
  f_1(n) + f_3(n) &\in O( \;\text{max}(f_2(n),f_4(n))) \qed
\end{align*}
\end{solution}

\begin{problem} (10 points) 
Let $f_1, f_2, f_3$ be functions from the set $\N$ of natural numbers
to the set $\R$ of real numbers. Suppose that $f_1= O(f_2)$ and
$f_2=O(f_3)$. Is it possible that 
$$ f_1(n) > f_3(n)$$ 
holds for all natural numbers $n$? Give an example or given an argument that this
is impossible. 
\end{problem}
\begin{solution}
Allow $f_3 = f_2 +1$ and $f_2 = f_1 +1$, where $f_1 = O(f_2)$ and $f_2 = O(f_3)$ is still true.
\begin{align*}
  f_3 &= f_2 +1 \\
  f_3 &= f_1 + 1 + 1 \\
  f_3 &= f_1 +2
\end{align*}
For all $n$, $f_3$ is 2 bigger than $f_1$. It then follows that $f_1 < f_3 \forall n \qed$
\end{solution}

\begin{problem} (5 pts $\times$ 4 = 20 points) 
Determine whether each of the following statements is true or false.
In each case, answer true or false, and justify your answer.
\begin{enumerate}[a)]
\item $3n^2-42 = O(n^2)$
\begin{align*}
  3n^2-42 &\le U n^{2} \\
  &\le 3n^{2} + n^{2} \\
  &\le 4 n^{2} &\text{for n $>$ 1}
\end{align*}
$\therefore 3n^2-42$ is $O(n^{2})$ with $U = 4$ and $n > 1$ as witnesses $\qed$

\item $n^2 = O(n\log n)$
\begin{align*}
  \dfrac{n^2}{n\log n} = \dfrac{n}{\log n}
\end{align*}
As $n$ grows, this does not approach nor is there $C$ that satisfies this equation for all $n$
$\therefore n^2$ is NOT $O(n\log n)$

\item $1/n = O(1)$
\begin{align*}
  \dfrac{1}{n} &\le U 1 \\
  \dfrac{1}{n} &\le 1 &\text{for n $>$ 1}
\end{align*}
$\therefore \dfrac{1}{n}$ is $O(1)$ with $U = 1$ and $n > 1$ as witnesses $\qed$

\item $n^n = \Omega(2^n)$
\end{enumerate}
\end{problem}
\begin{solution}
\end{solution}

\begin{problem} (15 points) 
Does $\Theta(n^3+2n+1) = \Theta(n^3)$ hold?  Justify your answer.
\end{problem}
\begin{solution}\ \\
A function $f(n)$ is $\Theta(g(n))$ iff \\ 
\hspace{10pt} $f(n) \le U g(n)$ for all $n>n_0$ \textit{$O$ definition} \\
and\\
\hspace{10pt} $f(n) \ge L g(n)$ for all $n>n_0$ \textit{$\Omega$ definition} \\
First, check if $O(n^3+2n+1) = O(n^3)$ holds
\begin{align*}
    n^3+2n+1 &\le U n^3 \\
    &\le n^3 + 2 n^3 + n^3\\
    &\le 4n^3
\end{align*}
$\therefore O(n^3+2n+1) = O(n^3)$ with $U=4$ and $n \ge 1$ as witnesses \\
Next, check $\Omega(n^3+2n+1) = \Omega(n^3)$
\begin{align*}
    n^3+2n+1 &\ge L n^3 \\
    n^3+2n+1 &\ge n^3 &\text{for $n \ge 1$}\\
    1 + \dfrac{2}{n^2} + \dfrac{1}{n^3} &\ge 1\\
    1 + \dfrac{2}{1} + \dfrac{1}{1} &\ge 1&\text{for $n \ge 1$}
\end{align*}
$\therefore O(n^3+2n+1) = O(n^3)$ \\
$\therefore \Omega(n^3+2n+1) = \Omega(n^3)$\\
$\therefore \Theta(n^3+2n+1) = \Theta(n^3) \qed$
\end{solution}

\begin{problem} (10 points)
Let $k$ be a fixed positive integer. Show that 
$$ 1^k+2^k+\cdots + n^k  = O(n^{k+1}) $$ holds.
\end{problem}
\begin{solution}
\begin{align*}
  1^k+2^k+\cdots + n^k  &\le U n^{k+1} \\
  &\le n^{k} + n^{k} + \cdots  + n^{k} \\
  &\le n \cdot n^{k} \\
  &\le 1 \cdot n^{k+1} &\text{for n $>$ 1}
\end{align*}
$\therefore 1^k+2^k+\cdots + n^k$ is $O(n^{k+1})$ with $U = 1$ and $n > 1$ as witnesses $\qed$
\end{solution}

\begin{problem} (15 points)
Suppose that you have two algorithms $A$ and $B$ that solve the same
problem. Algorithm $A$ has worst case running time $T_A(n) = 2n^2-2n+1$ 
and Algorithm $B$ has worst case running time $T_B(n) = n^2+n-1$.
\end{problem}
\begin{solution}\ \\
\begin{enumerate}[a)]
\item Show that both $T_A(n)$ and $T_B(n)$ are in $O(n^2)$.
\begin{align*}
  T_A(n) = 2n^2 - 2n + 1 &\le U n^2 \\
  &\le 2n^2 + n^2 + n^2 \\
  &\le 4 n^2 &\text{for $n > 1$}
\end{align*}
$\therefore T_A(n)$ is $O(n^2)$ with $U = 4$ and $n>1$ as witnesses \\

\begin{align*}
  T_B(n) = n^2+n-1 &\le U n^2 \\
  &\le 2n^2 + n^2 + n^2 \\
  &\le 4 n^2 &\text{for $n > 1$}
\end{align*}
$\therefore T_B(n)$ is $O(n^2)$ with $U = 4$ and $n>1$ as witnesses

\item Show that $T_A(n) = 2n^2 + O(n)$ and $T_B(n) = n^2 +O(n)$.
\begin{align*}
  -2n+1 &\le Un \\
  &\le n + n \\
  &\le 2n \\
  -2n+1 &\in O(n)
\end{align*}
$\therefore T_A(n) = 2n^2-2n+1=2n^2 + O(n)$
\begin{align*}
  n-1 &\le Un \\
  &\le n + n \\
  &\le 2n \\
  n-1 &\in O(n)
\end{align*}
$\therefore T_A(n) = n^2+n-1=n^2 + O(n)$

\item Explain which algorithm is preferable. \\
For large $n \; T_B(n)$ is preferable because due to the fact that the coefficient of $n^2$ is 1 instead of 2, like in $T_A(n)$
\end{enumerate}
\end{solution}

\begin{problem} (10 points) Section 3.3, Exercise 14 on page 230.
\end{problem}
\begin{solution}\ \\
\begin{enumerate}[a)]
  \item \ \\
    $n=2\; a_2=3 \; a_1=1 \; a_0=1$ \\
    Line 1: $y:= 3$ \\
    For Loop: \\
    $i:=1 \quad y:= 3*2 + 1 => 7$ \\
    $i:=2 \quad y:= 7*2 + 1 => 15 $ \\
    return 15
  \item \ \\
  There is one multiplication and one addition within the for loop. The loop executes $n$ times for $x=n$. \\
  $\therefore$ there are $n$ additions and $n$ multiplications 
\end{enumerate}
\end{solution}

\begin{problem} (10 points) Section 3.3, Exercise 16 a), d), g) and h) on page 230.
\end{problem}
\begin{solution}\ \\
Seconds in a day $= 10^5 sec$ \\
\# of Operations $= 10^{16}$ \\
a) \\
  \begin{align*}
    f(n) &\le 10^{16} \\
    \log n &\le 10^{16} \\
    n &\le 2^{10^{16}}
  \end{align*}
  Largest value of $n$ is $2^{10^{16}}$ \\
d) \\
  \begin{align*}
    f(n) &\le 10^{16} \\
    1000 n^2 n &\le 10^{16} \\
    n^2 &\le 10^{13} \\
    n &\le 10^{13/2}
  \end{align*}
  Largest value of $n$ is $10^{13/2}$ \\
g) \\
  \begin{align*}
    f(n) &\le 10^{16} \\
    2^{2n} &\le 10^{16} \\
    2n &\le \log_2 {10^{16}} \\
    n &\le \dfrac{1}{2}  \log_2 {10^{16}}
  \end{align*}
  Largest value of $n$ is $\dfrac{1}{2}  \log_2 {10^{16}}$ \\
h) \\
  \begin{align*}
    f(n) &\le 10^{16} \\
    2^{2^2n} &\le 10^{16} \\
    2^n &\le \log_2 {10^{16}} \\
    n &\le \log_2(\log_2 {10^{16}})
  \end{align*}
  Largest value of $n$ is $\log_2(\log_2 {10^{16}})$ \\
\end{solution}

\goodbreak
\checklist
\end{document}
